\chapter{PENGERTIAN BILANGAN BINER ATAU BINARY}

Bilangan biner atau bisa juga disebut bilangan binary merupakan sistem penulisan angka dengan hanya menggunkan dua simbol yaitu 1 dan 2. bilangan biner merupakan dasardari semua sistem bilangan yang berbasis digital. dari sistem biner kita dapat mengkonversikannya ke sistem bilangan Oktal atau Hexadesimal.

Bilangan biner umumnya digunakan dalam dunia komputasi. komputer menggunakan bilangan biner agar dapat saling berinteraksi terhadap semua komponen (hardware) dan bisa juga berinteraksi terhadap sesama komputer. contoh nya pada sebuah komputer yaitu apabila sebuah komputer terhubung dengan tegangan listrik maka bernilai 1 dan apabila komputer tidak terhubung dengan jaringan listrik makanilai nya 0.

\section{PENGURANGAN BILANGAN BINER}

keadaan yang muncul dipengurangan bil biner
(0-0, 1-0, 1-1)

Dimana pada keadaan

	0-0 = 0
	0-1 = 1 meminjam 1 (apabila masih ada angka disebelah kirinya)
	1-0 = 1
	1-1 = 0
	
maksud dari peminjaman angka adalah meminjam ketika angka yang disebelah kiri lebih besar agar dapat mencukupi ketika melakukan pengurangan.

keadaan yang sama pun akan berlaku pada bilangan desimal.
contohnya pada bilangan desimal:

	45-40 = 5 (tidak perlu meminjam karena nilainya mencukupi)
pengurangan yang satu ini tidak perlu melakukan peminjaman angka karena angkanya sudah mencukupi dalam melakukan pengurangan
				
	20-19 = 1 (angka 0 harus meminjam 1 dari angka 2 supaya dapat menguranginya)
		caranya: ketika angka 0 meminjam 1 angka dari angka 2 maka nilainya menjadi 10-9.
