\section{Arduino}
\subsection{Pengertian}
Arduino Uno merupakan board mikrokontroler yang memiliki basis ATM 328 atau datasheet. Arduino Uno memiliki 14 pin input dari output digital yang 6 pin input diantarnya dapat digunakan untuk outuot PWM dan 6 input lagi untuk analog, 16 MHz osilator kristal, jack power, tombol reset, ICSP header, dan koneksi USB. Untuk memaksimalkan kerja mikrokontroler agar dapat digunakan dengan maksimal sangat mudah caranya, hanya perlu menghubungkan Board Arduino Uno ke komputer menggunakan listrik dengan  AC yang ke arah  adaptor DC atau menggunakan kabel USB atau  juga bisa dengan menggunakan baterai untuk menjalankan Board Arduino Uno tersebut dan masih banyak cara lainnya.
Arduino Uno memiliki perbedaan dengan semua board yang telah diciptakan sebelumnya dalam hal koneksi USB to serial, yang dimaksud dengan koneksi USB to serial adalah menggunakan fitur Atmega 8U2 yang memiliki program sebagai konveter USB to serial yang berbeda dengan board yag telah diciptakan sebelumnya, pada board sebelumnya menggunakan chip FTDI driver USB to serial.
Uno berasal dari bahasa italia yang memiliki arti satu, hal ini digunakan untuk menandani peluncuran Arduino 1.0. Uno dan versi 1.0 yang akan menjadi versi referensi dari Arduino tersebut. Uno adalah seri terbaru dalam serangkaian board USB Arduino sekaligus menjadi model platform Arduino, dan menjadi perbandingan dengan versi Arduino sebelumnya.
\subsection{Konsep Arduino}
Papan Arduino juga memiliki tugas sebagai I / O pada tingkat rendah yang diperlukan untuk mengatur yang tidak konvensional  pada perangkat keras tambahan, memerlukan beberapa solusi kreatif yang dapat membantu, dan untuk keterampilan pemograman. Perhitungkan, contohnya, daftar singkat pada proyek terbuka yang mungkin akan sangat berguna di dalam laboratorium psikologis dan neurofisiologis. Sensor sentuhan yang sangat mudah untuk dibangun dan murah harganya telah dirancang untuk digunakan dalam penelitian perilaku yang melibatkan pengukuran waktu reaksi atau disingkat RT peserta yang telah mencapai dan menyentuh benda tersebut. Proyek lain menunjukkan bagaimana cara membangun fotodioda yang dapat digunakan, misalnya, untuk mendeteksi presentasi stimuli pada layar.