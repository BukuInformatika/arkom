\section{Definisi arduino}
Arduino adalah mikrokontroler singleboard opensource, berasal dari platform Wiring, yang telah disetting untuk memudahkan penggunaan elektronik di berbagai bidang. Perangkat kerasnya memiliki prosesor Atmel AVR dan perangkat lunaknya memiliki bahasa pemrograman tersendiri. Bahasa pemrograman yang digunakan arduino adalah bahasa pemrograman C atau C ++, hal ini dimaksudkan untuk memudahkan arduino untuk membaca codingan yang dibuat.


Arduino juga merupakan platform perangkat keras terbuka yang dapat digunakan untuk siapa saja yang ingin membuat prototip peralatan elektronik interaktif berdasarkan perangkat keras dan perangkat lunak yang fleksibel dan mudah digunakan. Mikrokontroler diprogram menggunakan bahasa pemrograman arduino yang memiliki kesamaan sintaksis dengan bahasa pemrograman C Karena sifatnya yang terbuka maka siapapun bisa mendownload skema hardware arduino dan membangunnya.

Arduino menggunakan keluarga mikrokontroler ATMega yang diluncurkan oleh Atmel, namun banyak perusahaan membuat buatan artifisial menggunakan mikrokontroler lainnya dan tetap kompatibel dengan arduino di tingkat perangkat keras. Untuk fleksibilitas, program ini dimuat melalui bootloader meski ada pilihan untuk bypass pada bootloader dan menggunakan downloader untuk memprogram mikrokontroler secara langsung melalui port ISP.

\subsection{Konsep arduino}
	Sumber Daya listrik. Papan pemrosesan Arduino mungkin didukung dari port USB selama pengembangan proyek.Namun, sangat disarankan agar catu daya eksternal dipekerjakan. Ini akan memungkinkan pengembangan proyek di luar kemampuan arus port USB yang terbatas. www.arduino.cc merekomendasikan catu daya dari 7-12 VDC dengan konektor positif pusat 2,1 mm. hal ini bertujuan agar daya yang terkonek pada arduino lebih stabil

	Catu daya jenis ini sudah tersedia dari sejumlah perusahaan pemasok komponen elektronik.Misalnya, catu daya jameco 133891 adalah model 9 VDC yang diberi nilai 300 mA dan dilengkapi dengan konektor positif pusat 2,1 mm. Tersedia dengan harga di bawah 10 US dollar, catu daya ini memiliki harga yang tidak terlalu mahal sehingga sebagian besar orang menggunakan produk tersebut.