\section{Arduino}
\subsection{Sejarah Arduino}
	Sejarah Arduino berawal dari sebuah tesis yang dibuat oleh seseorang bernama Hernando Baragan, di institute of Ivrea, Italia pada tahun 2005, kemudian dikembangkan oleh Massimo Banzi dan David Cuartielles dan diberi nama Arduin of Ivrea. Kemudian diganti nama menjadi Arduino yang dalam bahasa Italia berarti teman yang berani.

	Tujuan awal dibuat Arduino adalah untuk membuat suatu perangkat mudah dan harganya murah, dari perangkat yang ada saat itu. Dan perangkat ini akan ditujukan kepada para siswa yang akan membuat perangkat desain dan interaksi.

	Tim pengembangnya pada saat ini adalah Gianluca Martino, David Cuartielles, Tom Igoe, David Mellis, Massimo Banzi, dan Nicholas Zambetti. Mereka mengupayakan 4 hal dalam Arduino ini, yaitu:
	1.	Harga yang terjangkau
	2.	Dapat dijalankan di berbagai sistem operasi(OS). Seperti: Windows, Linux, Max, dan sebagainya.
	3.	Sederhana, menggunakan bahasa pemograman yang mudah bisa dipelajari orang awam, bukan hanya untuk orang teknik saja.
	4.	Open Source, baik hardware maupun software.

	Sifat yang dimiliki Arduino adalah Open Source, sehingga membuat Arduino berkembang dengan sangat cepat dan banyak dibuat perangkat-perangkat sejenis Arduino. Seperti DFRDuino atau Freeduino,  yang lokal namanya CipaDuino yang dibuat oleh SKIR70, MurmerDuino yang dibuat oleh Robot Unyil, ada lagi AViShaDuino yang salah satu pembuatnya adalah Admin Kelas Robot.
	
	Saat ini pihak resmi telah membuat berbagai jenis Arduino.  paling banyak digunakan yaitu Arduino Uno, Hingga Arduino yang telah menggunakan ARM Cortex, beebentuk Mini PC. Sudah ada ratusan ribu Arduino yang digunakan di gunakan di dunia hingga tahun ini. 

\subsection{Konsep Arduino}
	Perangkat lunak ini terdiri dari bahasa pemrograman standar dan firmware yang berjalan di papan tulis. Perangkat keras Arduino diprogram menggunakan bahasa yang disederhanakan C ++, dalam IDE berbasis pengolahan. Perangkat lunak ini kemudian disusun dan dimuat di kapal. Papan Arduino juga kompatibel dengan Flash, Processing, MaxMSP, dan MATLAB, dan beberapa baris kode seringkali cukup untuk memungkinkan perilaku yang cukup kuat