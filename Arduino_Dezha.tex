/section{Arduino}
a. Arduino Uno
 Arduino ini adalah jenis arduino yang paling banyak digunakan saat ini. Khususnya untuk pemula yang baru belajar menggunakan arduino sangat disarankan untuk menggunakan Arduino Uno. Saat ini banyak sekali referensi yang membantu untuk mempelajari Arduino Uno. Versi Arduino yang terakhir adalah Arduino Uno R3 (Revisi 3), menggunakan ATMEGA328 sebagai Microcontrollernya, dan memiliki 14 pin I/O digital dan 6 pin input analog.Untuk pemograman cukup menggunakan koneksi USB type A to To type B. Sama seperti yang digunakan pada USB printer.
b. Arduino Due
Berbeda dengan saudaranya, Arduino Due tidak menggunakan micro controller ATMEGA, melainkan dengan chip yang lebih tinggi ARM Cortex CPU. Memiliki 54 I/O pin digital dan 12 pin input analog.Untuk pemogramannya menggunakan Micro USB, terdapat pada beberapa handphone.
c. Arduino Mega Mirip dengan Arduino Uno, sama-sama menggunakan USB type A to B untuk pemogramannya. Tetapi Arduino Mega, menggunakan Chip yang lebih tinggi yaitu ATMEGA2560. Dan tentu saja untuk Pin I/O Digital dan pin input Analognya lebih banyak dari Uno
d. Arduino Leonardo. Bisa dibilang Leonardo adalah saudara kembar dari Uno. Dari mulai jumlah pin I/O digital dan pin input Analognya sama. Hanya pada Leonardo menggunakan Micro USB untuk pemogramannya.

/section{Konsep Arduino}
Namun, ketepatan waktu memiliki prioritas tertinggi untuk semua eksperimentalis, meski harganya mahal.