\section{Jenis-jenis arduino}
	Terdapat beberapa jenis arduino jenis yang pertama adalah Arduino Fio, Arduino Fio memiliki bentuk yang lebih unik, terutama pada bagian socketnya. Meskipun mempunyai jumlah pin I/O digital dan juga input analognya sama dengan uno maupun leonardo,akan tetapi fio mempunyai socket XBeemembuat fio digunakanuntuk keperluan projek yang berhubungan dengan wireless. Selanjutnya Arduino Lilypad yaitu memiliki bentuk unik yaitu melingkar, lilypad dapat dipakai untuk membuat projek-projek yang unik tetapi juga dapat membuat suatu projek yang sangat keren.
	Dengan 14 pinpin I/O digital, dan juga 6 pin input analognya.


	Yang berikutnya merupakan arduino Nano seperti halnya dengan namanya "nano" yang memiliki arti ukuran  yang sangat kecil dan juga sangat sederhana ini, menyimpan banyak sekali fasilitas, dan juga sudah dilengkapi dengan FTDI untuk pemrograman yang melalui micro USB.Dengan dilengkapi 14 pin I/O digital, dan juga 8 pin input analog yang memiliki lebih banyak dari pada UNO. Dan ada juga yang menggunakan ATMEGA168 atau pun menggunakan ATMEGA328.

	Untuk yang selanjutnya adalah jenis Arduino Mini dan juga jenis Arduino Micro.Arduino Mini memiliki fasilitas yang sama seperti yang dimiliki Arduino Nano yaitu menyimpan banyak sekali fasilitas, akan tetapi Arduino jenis ini tidak dilengkapi dengan Micro USB untuk pemrograman. Dan hanya memiliki ukuran 30mm x 18mm saja.
	Arduino Micro, Arduino jenis ini memiliki ukuran lebih panjang dari pada Arduino Nano dan juga Arduino Mini. Karena memang memiliki fasilitas yang lebih banyak yaitu memiliki 20 pin I/O digital dan juga 12 pin input analog.

\subsection{contoh projek arduino}
	a.Arduino Lilypad contoh membuat Amor iron man dengan menggunakan versi ATMEGA168.
