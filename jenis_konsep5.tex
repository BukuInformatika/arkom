\section{Jenis-jenis arduino}
	Terdapat beberapa jenis arduino jenis yang pertama adalah Arduino Fio, Arduino Fio memiliki bentuk yang lebih unik, terutama pada bagian socketnya. Meskipun mempunyai jumlah pin I/O digital dan juga input analognya sama dengan uno maupun leonardo,akan tetapi fio mempunyai socket XBeemembuat fio digunakanuntuk keperluan projek yang berhubungan dengan wireless. Selanjutnya Arduino Lilypad yaitu memiliki bentuk unik yaitu melingkar, lilypad dapat dipakai untuk membuat projek-projek yang unik tetapi juga dapat membuat suatu projek yang sangat keren.
	Dengan 14 pinpin I/O digital, dan juga 6 pin input analognya.


	Yang berikutnya merupakan arduino Nano seperti halnya dengan namanya "nano" yang memiliki arti ukuran  yang sangat kecil dan juga sangat sederhana ini, menyimpan banyak sekali fasilitas, dan juga sudah dilengkapi dengan FTDI untuk pemrograman yang melalui micro USB.Dengan dilengkapi 14 pin I/O digital, dan juga 8 pin input analog yang memiliki lebih banyak dari pada UNO. Dan ada juga yang menggunakan ATMEGA168 atau pun menggunakan ATMEGA328.

	Untuk yang selanjutnya adalah jenis Arduino Mini dan juga jenis Arduino Micro.Arduino Mini memiliki fasilitas yang sama seperti yang dimiliki Arduino Nano yaitu menyimpan banyak sekali fasilitas, akan tetapi Arduino jenis ini tidak dilengkapi dengan Micro USB untuk pemrograman. Dan hanya memiliki ukuran 30mm x 18mm saja.
	Arduino Micro, Arduino jenis ini memiliki ukuran lebih panjang dari pada Arduino Nano dan juga Arduino Mini. Karena memang memiliki fasilitas yang lebih banyak yaitu memiliki 20 pin I/O digital dan juga 12 pin input analog.


	Yang berikutnya adalah Arduino Ethernet dan Arduino Esplora. Arduino Ethernet, jenis arduino ini yang menyediakan beberapa cara dan beberapa protokol seperti HTTP, TCP, dan UDP yang memungkinkan terjadi program sketch yang menjadikan arduino sengai client atau server. Penjelasan client dan server merupakan software yang bentuk fisiknya dapat bermacam-macam seperti PC, mainframe, laptop dan microcontroller. Dan membuat arduino kamu dapat berhubungan langsung dengan LAN yang ada pada komputer. Untuk fasilitas pada pin I/O digital dan memiliki input analog yang sama dengan Uno. 
	Ke dua yaitu Arduino Eksplora merupakan rekomendasi bagi kamu yang ingin membuat gadget seperti smartphone, karena sudah dilengkapi dengan LCD yang berfungsi untuk lebih mempercantik eksploranya.

	Berikutnya merupakan Arduino Robot, arduini ini adalah paket yang lengkap dari arduino yang telah berbentuk robot. Dan juga telah dilengkapi dengan LCD, Speaker, Roda, Sensor Infrared dan juga semua yang kita butuhkan untuk pembuatan robot sudah ada di dalam arduino ini.

\subsection{contoh projek arduino}
	a.Arduino Lilypad contoh membuat Amor iron man dengan menggunakan versi ATMEGA168.

	b.Arduino Nano contoh membuat lampu LED
	c. Arduino Uno contoh membuat Alarm suara , Timer sleepy , MP3 Player, Game controller,


\section{ Konsep Arduino}
	Perangkat keras Arduino terdiri atas desain-desain hardware yang terbuka dengan prosesor Atmel AVR. Papan arduino dapat dibeli dengan pressembled, namun informasi dari desain perangkat keras juga tersedia untuk mereka yang ingin membangun ataupun memodifikasi. Adan beberapa pembuat pihak ketiga telah memproduksi Shield add-on board yang mampu memperluas kemampuan dasar dari arduino. Motor Control Shield memungkinkan untuk control motor DC dan pembentuk kode baca , perisai Xbee memungkinkan beberapa papan arduino untuk berkomunikasi secara nirkabel, dan Shield Velocity Accelorometer kritis menyatukan akselerometer 3 sumbu.  


\subsection(Variasi Konsep Arduino)
	a. Pihak ketiga telah merilis beberapa variasi dari konsep arduino, ini merupakan papan bangunan perusahaan yang biasanya menggunakan spesifikasi lebih baik namun harga lebih rendah dengan menggunakan perangkat arduino