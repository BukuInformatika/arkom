Format standar :
1. untuk list dan nomor gunakan enumerate atau itemize
	contoh :
	berikut nama anggota kelompok
	\begin{enumerate}
	\item darso
	\item karyo
	\item doyok
	\end{enumerate}
	
	\begin{enumerate}
	\item
	This is the first item in the numbered list.

	\item
	This is the second item in the numbered list.
	\end{enumerate}

	\begin{itemize}
	\item
	This is the first item in the itemized list.

	\item
	This is the first item in the itemized list.
	This is the first item in the itemized list.
	This is the first item in the itemized list.
	\end{itemize}

	\begin{itemize}
	\item[]
	This is the first item in the itemized list.

	\item[]
	This is the first item in the itemized list.
	This is the first item in the itemized list.
	This is the first item in the itemized list.
	\end{itemize}

2. spesial karakter menggunakan tanda \ didepannya
	contoh :
	\&
	\_
	\"dalam petik\"
	jika spesial karakter menjadi banyak atau satu baris gunakan verb
	contoh :
	\verb|%$'%&$&'%'%'%&'%|
	
3. untuk tabel gunakan table , contoh

\begin{table}[h]
\caption{Small Table}
\centering
\begin{tabular}{ccc}
\hline
one&two&three\\
\hline
C&D&E\\
\hline
\end{tabular}
\end{table}

4. untuk rumus gunakan tag equation
	contoh:
	Rumus bola:

	a) Luas permukaan
	 \begin{equation}
	     L = 4 \pi r^2 \,
	\end{equation}
	b) Volume
	 \begin{equation}
	     V = \frac{4}{3}\pi r^3
	\end{equation}
	
5. 	untuk kode program menggunakan verbatim
	\begin{verbatim}
	a = "anu"
	b = "itu"
	c = a + b
	print(c) 
	\end{verbatim}	
6. bab subbab subsubbab :
	\section{Bab}
	\subsection{Sub Bab}
	\subsubsection{Sub Sub Bab}
	\paragraph{paragraph}

7. Tabel 
-----
Tables:
 Remember to use \centering for a small table and to start the table
 with \hline, use \hline underneath the column headers and at the end of 
 the table, i.e.,

\begin{table}[h]
\caption{Small Table}
\centering
\begin{tabular}{ccc}
\hline
one&two&three\\
\hline
C&D&E\\
\hline
\end{tabular}
\end{table}

For a table that expands to the width of the page, write

\begin{table}
\begin{tabular*}{\textwidth}{@{\extracolsep{\fill}}lcc}
\hline
....
\end{tabular*}
%% Sample table notes:
\begin{tablenotes}
$^a$Refs.~19 and 20.

$^b\kappa, \lambda>1$.
\end{tablenotes}
\end{table}


8. untuk quote, Sample quote:
\begin{quote}
quotation...
\end{quote}

9 Untuk algoritma
Algorithm.
Maintains same fonts as text (as opposed to verbatim which uses fixed
width fonts). Space at beginning of line will be maintained if you
use \ at beginning of line.

\begin{algorithm}
{\bf state\_transition algorithm} $\{$
\        for each neuron $j\in\{0,1,\ldots,M-1\}$
\        $\{$   
\            calculate the weighted sum $S_j$ using Eq. (6);
\            if ($S_j>t_j$)
\                    $\{$turn ON neuron; $Y_1=+1\}$   
\            else if ($S_j<t_j$)
\                    $\{$turn OFF neuron; $Y_1=-1\}$   
\            else
\                    $\{$no change in neuron state; $y_j$ remains %
unchanged;$\}$ .
\        $\}$   
$\}$   
\end{algorithm}

-----




