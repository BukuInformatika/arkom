Sejarah Perkembangan Wi-Fi (Wireless Fidelity)
Wireless Fidelity adalah satu standart wireless networking tanpa kabel. teknologi spesifikasi ini emiliki standart yang ditetapkan oleuh sebuah institusi internasional yang bernama IEEE  (Insitute of Electrical and Electronic Engineers). Di tahun 1997 sebuah lembagaindependen bernama IEEE membuat standart WLAN pertama yang diberi kode 802.11. dapat bekerja pada frekuensi 2,4GHz dengan kecepatan transfer data 2Mbps.
Empatsejarah singkat perkembangan protokol Wireless fidelity:
1. pada bulan juli 1999, IEEE merilis spesifikasi baru yang bernama 802.11b. dengan kecepatan transfer data maksimal11Mbps.
2. pada waktu yang hampir bersamaan institute of electrical and electronic engineers kembali membuat spesifikasi 802.11a yang menggunakan teknik berbeda. Frekuensi yang  digunakan 5GHz, dan mendukung kecepatan transfer data dengan kecepatan sampai 54Mbps
3.pada tahun 2002. institute of electrical and electronic engineers membuat spesifikasi baru yang dapat menggabungkan kelebihan antara 802.11b dengan 802.11a. spesifikasi baru  yang diberi kode 802.11g ini bekerja pada frekuensi 2,4GHz dengan kecpatan transfer data maksimal 54Mbps.
4.di tahun 2006 institute of electrical and electronic engineers mengembangkan teknologi terbarunya dengan menggabungkan teknologi 802.11b dengan 802.11g menjadi 802.11n. teknologi ini dikenal dengan istilah MIMO (Multiple Input Multiple Output) teknologi wireless fidelity terbaru
 
SEJARAH WI-FI
Sejarah Wi-Fi berasal dari sebuah istilah puluhan tahun Hi-Fi yang terdiri dari jenis output yang dihasilkan olehkualitas sound system. Teknollgi Wireless Fidelity berspesifikasi standart Institute of Electrical and Electronic Engineers atau yang disingkat dengan IEEE 802.. termasuk 802.11a, 802.11b, dan 802.11g. Wireless Fidelity atau yang disingkat dengan Wi-Fi adalah hanya istilah produk teknologi yang dipromosikan oleh WIFI Alliance.
Sejarah Wireless Fidelity itu sendiri dimulai ketika tahun 1985 dari hasil kerja keras insinyur Amerika dengan pengguna Teknologi penyebaran spektrum radio yang digunakan dalam Wi-Fi. Wireless LAN atau Wi-Fi dibuat dan tersedia untuk umum di Amerika Serikat di tahun 1985, tidak ada lisensi dari komisi komunikasi federal (FCC). Kemudian Michael Marcus mengusulkan untuk menggunakan wireless LAN dan teknologi radio untuk publik.

Wi-Fi adalah sebuah teknologi yang memanfaatkan peralatan teknologi untuk bertukar data menggunakan gelombang radio melalui jaringan komputer. Vic Hayes adalah penemu Wi-Fi yang kini dijuluki sebagai “ Father of Wi-Fi “. WI-Fi merupakan sekumpulan standar yang digunakan untuk Jaringan Lokal Nirkabel yang memiliki spesifikasi IEEE 802.11. Pengertian dari IEEE tersebut adalah sebuah organisasi internasional yang mempublikasikan beberapa persoalan kunci dari dunia networking komputer. Ada awalnya Wi-Fi hanya digunakan pada jaringan Lokal (LAN),seiring berjalannya waktu Wi-Fi dimanfaatkan masyarakat untuk mengakses internet. Penerapan Wi-Fi  ditujukan sebagai alternatif dari jaringan Lokal komputer LAN,dimana penggunaan kabel sudah tidak lagi effisien. Wi-Fi memiliki mobilitas yang tinggi,sehingga untuk mengakses WI-Fi ini tidak diperlukannya penyambung kabel untuk menghubungkan ke server.
Pada dasarnya,Wi-Fi terdiri dari sumber yang dihubungkan dengan access point melalui kabel backbone. Selanjutnya dipancarkan melalui gelombang elektromagnetik seperti pada LAN kabel biasa yang kemudian diterima oleh client (Contohnya PC desktop) melalui wireless adapter yang mendukung jaringan Wi-Fi berdasarkan standarisasi IEEE 802.11. Tetapi access point ini memiliki area yang sangat terbatas,500 feet (152.4 M) dalam ruangan tertutup dan 1000 feet (304.8 M) dalam ruangan terbuka.
Wi-Fi akan mengalami proses handoffs agar wireless client dapat melanjutkan komunikasi dengan server yang berbeda. Wireless client akan terus memonitor sinyal yang diterima oleh access point,jika kuat sinyal kurang dari nilai sensitivitas penerimaan (threshold) maka wireless akan melakukan handoffs yang selanjutnya akan mencari sinyal terdekat. Proses identifikasi dari wireless client untuk menemukan sinyal access point terkuat hanya dibatasi dalam waktu 60 second. Backbone search time adalah proses pencarian AP dan EP untuk dijadikan BSS. Untuk dapat berkomunikasi yang lama antara wireless client dengan access point harus memiliki level daya yang diterima di atas -77 dBm,jika kurang dari -77 dBm maka wireless client akan melakukan proses handoffs dengan beralih pada daya yang lebih tinggi dari access point sebelumnya.
Dibalik kelebihannya Wi-Fi yang sudah memiliki kebutuhan  akses internet yang lebih baik dibandingkan dengan akses internet yang menggunakan kabel,tetapi Wi-Fi masih memiliki beberapa kekurangan sekarang ini,diantaranya ada :
	Area coverage-nya yang sangat sempit,hanya dalam hitungan meter
	Hanya mencukupi akses internet dalam suatu daerah atau dalam ruangan saja
	Keamanan yang belum terjamin
	Membutuhkan banyak BTS untuk menjangkau seluruh area yang luas
	LoS (Line of Sight) 

 Cara Kerja Wi-Fi
 Mode Akses Koneksi Wi-fi ada 2 yaitu :
1. AD-HOC
Sistem Ad-hoc adalah sistem peer to peer, dalam arti satu computer dihubungkan ke 1 computer dengan saling mengenal SSID. Bila digambarkan mungkin lebih mudah membayangkan sistem direct connection dari 1 computer ke 1 computer lainnya dengan mengunakan Twist pair cable tanpa perangkat HUB. Jadi terdapat 2 computer dengan perangkat WIFI dapat langsung berhubungan tanpa alat yang disebut access point mode. Pada sistem Adhoc tidak lagi mengenal system central (yang biasanya difungsikan pada Access Point). Sistem Adhoc hanya memerlukan 1 buah computer yang memiliki nama SSID atau sederhananya nama sebuah network pada sebuah card/computer. Dapat juga mengunakan MAC address dengan sistem BSSID (Basic Service Set IDentifier - cara ini tidak umum digunakan), untuk mengenal sebuah nama computer secara langsung. Mac Address umumnya sudah diberikan tanda atau nomor khusus tersendiri dari masing masing card atau 
perangkat network termasuk network wireless. Sistem Adhoc menguntungkan untuk pemakaian sementara misalnya hubungan network antara 2 computer walaupun disekitarnya terdapat sebuah alat Access Point yang sedang bekerja.
2. INFRASTRUKTUR
Sistem kedua yang paling umum adalah Infra Structure. Sistem Infra Structure membutuhkan sebuah perangkat khusus atau dapat difungsikan sebagai Access point melalui software bila mengunakan jenis Wireless Network dengan perangkat PCI card. Mirip seperti Hub Network yang menyatukan sebuah network tetapi didalam perangkat Access Point menandakan sebuah sebuah central network dengan memberikan signal (melakukan broadcast) radio untuk diterima oleh computer lain. Untuk mengambarkan koneksi pada Infra Structure dengan Access poin minimal sebuah jaringan wireless network memiliki satu titik pada sebuah tempat dimana computer lain yang mencari menerima signal untuk masuknya kedalam network agar saling berhubungan. Sistem Access Point (AP) ini  paling banyak digunakan karena setiap computer yang ingin terhubungan kedalam network dapat mendengar transmisi dari Access Point tersebut. Access Point inilah yang memberikan
 tanda apakah disuatu tempat memiliki jaringan WIFI dan secara terus menerus mentransmisikan namanya – Service Set IDentifier (SSID) dan dapat diterima oleh computer lain untuk dikenal. Bedanya dengan HUB network cable, HUB mengunakan cable tetapi tidak memiliki nama (SSID). Sedangkan Access point tidak mengunakan cable network tetapi harus memiliki sebuah nama yaitu nama untuk SSID. Contoh Wi-fi Hardware yang digunakan di masyarakat : Wi-fi dalam bentuk PCI Wi-fi dalam bentuk USB



