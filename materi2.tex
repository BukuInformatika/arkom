\chapter{Managemen Perangkat Lunak Input/Output}
Perangkat Input/Output berdasarkan sasaran komunikasi dapat dibagi menjadi 3 bagian, antara lain 
\begin{enumerate} 
\item Perangkat yang terbaca oleh manusia.
   Perangkat yang cocok untuk komunikasi dengan manusia
   Contoh : VDT (Video Displayy Terminal) terdiri dari monitor, keyboard,dan mungkin 
   ditambah mouse.
\item Perangkat yang terbaca oleh mesin
   Perangkat yang cocok untuk komunikasi dengan perangkat elektronik.
   Contoh : Disk, tape, sensor, controller,dan aktuator.

\item Untuk komunikasi
   Perangkat yang cocok untuk komunikasi dengan perangkat jarak jauh.
   Contoh : modem.
=======
\chapter{Managemen Perangkat Lunak Input/Output}
Perangkat Input/Output berdasarkan sasaran komunikasi dapat dibagi menjadi 3 bagian, antara lain 
\begin{enumerate} 
\item Perangkat yang terbaca oleh manusia.
   Perangkat yang cocok untuk komunikasi dengan manusia
   Contoh : VDT (Video Displayy Terminal) terdiri dari monitor, keyboard,dan mungkin 
   ditambah mouse.
\item Perangkat yang terbaca oleh mesin
   Perangkat yang cocok untuk komunikasi dengan perangkat elektronik.
   Contoh : Disk, tape, sensor, controller,dan aktuator.

\item Untuk komunikasi
   Perangkat yang cocok untuk komunikasi dengan perangkat jarak jauh.
   Contoh : modem.
>>>>>>> 1b7b3006b1c518b467f4dae72c8795e83253a4fc
\end{enumerate}