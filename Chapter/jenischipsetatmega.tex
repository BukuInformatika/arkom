\section{Mikrokontroler ATmega8535}

	\subsection{Mikrokontroler}
		Mikrokontroler merupakan suatu komponen elektronika yang didalamnya terdapat rangkaian mikroprosesor, memori (RAM/ROM) dan I/O, rangkaian tersebut terdapat dalam level chip atau biasa disebut single chip microcomputer. Pada mikrokontroler sudah terdapat komponen-komponen mikroprosesor dengan bus-bus internal yang saling berhubungan. Komponen–komponen tersebut adalah RAM, ROM, Timer, komponen I/O paralel dan serial, dan interrupt controller. Dengan harga yang terjangkau memungkinkan mikrokontroler digunakan pada berbagai sistem elektronis, seperti pada robot, sistem alarm, peralatan telekomunikasi, hingga sistem automasi industri.
		
	\subsection{Chipset ATMEGA 8535}
		Mikrokontroler sebagai sebuah “one chip solution” pada dasarnya adalah rangkaian terintregrasi (Integrated Circuit-IC) yang telah mengandung secara lengkap berbagai komponen pembentuk sebuah komputer. Berbeda dengan penggunaan mikroprosesor yang masih memerlukan komponen luar tambahan seperti RAM, ROM, Timer, dan sebagainya untuk sistem mikrokontroler, tambahan komponen diatas secara praktis hampir tidak dibutuhkan lagi. Hal ini disebabkan semua komponen penting tersebut telah ditanam bersama dengan sistem prosesor ke dalam IC tunggal mikrokontroler bersangkutan. Dengan alasan itu sistem mikrokontroler dikenal juga dengan istilah populer the real Computer On a Chip (komputer utuh dalam keping tunggal), sedangkan sistem mikroprosesor dikenal dengan istilah yang 	lebih terbatas yaitu Computer On a Chip (komputer dalam keping tunggal).
		Mikrokontroler AVR memiliki arsitektur RISC 8 bit, dimana semua instruksi dikemas dalam kode 16-bit (16-bits word) dan sebagian besar instruksi dieksekusi dalam 1 (satu) siklus clock, berbeda dengan instruksi MCS51 yang membutuhkan 12 siklus clock. Tentu saja itu terjadi karena kedua jenis mikrokontroler tersebut memiliki arsitektur yang berbeda. AVR berteknologi RISC (Reduced Instruction Set Computing), sedangkan seri MCS51 berteknologi CISC (Complex Instruction Set Computing). Secara umum, AVR dapat dikelompokkan menjadi empat kelas, yaitu keluarga ATtiny, keluarga AT90Sxx, keluarga ATMega dan AT86RFxx. Pada dasarnya yang membedakan masing-masing kelas adalah memori, peripheral dan fungsinya. Dari segi arsitektur dan instruksi yang digunakan, mereka bisa dikatakan hampir sama.

\section{ATmega8}

	\subsection{Penjelasan}
		Kali ini saya akan membahas lebih dalam tentang ATMega8. Disini saya fokuskan pada pembahasan tentang fungsi pin, clock, fuse bit, dll.
		Sedikit saya ulas tentang pembahasan saya pada tulisan sebelumnya, bahwa mikrokontroler ATMega8 merupakan mikrokontroler keluarga AVR 8bit. Beberapa tipe mikrokontroler yang “berkeluarga” sama dengan ATMega8 ini antara lain ATMega8535, ATMega16, ATMega32, ATmega328, dll. Yang membedakan antara mikrokontroler yang saya sebutkan tadi antara lain adalah, ukuran memori, banyaknya GPIO (pin input/output), peripherial (USART, timer, counter, dll).