\section{Mikrokontroler ATmega8535}

	\subsection{Mikrokontroler}
		Mikrokontroler merupakan suatu komponen elektronika yang didalamnya terdapat rangkaian mikroprosesor, memori (RAM/ROM) dan I/O, rangkaian tersebut terdapat dalam level chip atau biasa disebut single chip microcomputer. Pada mikrokontroler sudah terdapat komponen-komponen mikroprosesor dengan bus-bus internal yang saling berhubungan. Komponen–komponen tersebut adalah RAM, ROM, Timer, komponen I/O paralel dan serial, dan interrupt controller. Dengan harga yang terjangkau memungkinkan mikrokontroler digunakan pada berbagai sistem elektronis, seperti pada robot, sistem alarm, peralatan telekomunikasi, hingga sistem automasi industri.
		
	\subsection{Chipset ATMEGA 8535}
		Mikrokontroler sebagai sebuah “one chip solution” pada dasarnya adalah rangkaian terintregrasi (Integrated Circuit-IC) yang telah mengandung secara lengkap berbagai komponen pembentuk sebuah komputer. Berbeda dengan penggunaan mikroprosesor yang masih memerlukan komponen luar tambahan seperti RAM, ROM, Timer, dan sebagainya untuk sistem mikrokontroler, tambahan komponen diatas secara praktis hampir tidak dibutuhkan lagi. Hal ini disebabkan semua komponen penting tersebut telah ditanam bersama dengan sistem prosesor ke dalam IC tunggal mikrokontroler bersangkutan. Dengan alasan itu sistem mikrokontroler dikenal juga dengan istilah populer the real Computer On a Chip (komputer utuh dalam keping tunggal), sedangkan sistem mikroprosesor dikenal dengan istilah yang 	lebih terbatas yaitu Computer On a Chip (komputer dalam keping tunggal).
		Mikrokontroler AVR memiliki arsitektur RISC 8 bit, dimana semua instruksi dikemas dalam kode 16-bit (16-bits word) dan sebagian besar instruksi dieksekusi dalam 1 (satu) siklus clock, berbeda dengan instruksi MCS51 yang membutuhkan 12 siklus clock. Tentu saja itu terjadi karena kedua jenis mikrokontroler tersebut memiliki arsitektur yang berbeda. AVR berteknologi RISC (Reduced Instruction Set Computing), sedangkan seri MCS51 berteknologi CISC (Complex Instruction Set Computing). Secara umum, AVR dapat dikelompokkan menjadi empat kelas, yaitu keluarga ATtiny, keluarga AT90Sxx, keluarga ATMega dan AT86RFxx. Pada dasarnya yang membedakan masing-masing kelas adalah memori, peripheral dan fungsinya. Dari segi arsitektur dan instruksi yang digunakan, mereka bisa dikatakan hampir sama.
		
		\subsection{Konfigurasi Pin ATmega8535}
		Konfigurasi pin ATmega8535 bisa dilihat pada gambar 2.6, Dari gambar tersebut dapat dijelaskan secara fungsional konfigurasi pin ATmega8535 sebagai berikut :
		\begin{itemize}
			\item VCC merupakan pin yang berfungsi sebagai pin masukan catu daya
			\item GND merupakan pin ground.
			\item Port A (PA0..PA7) merupakan pin I/O dua arah dan pin masukan ADC
			\item Port A (PA0..PA7) merupakan pin I/O dua arah dan pin masukan ADC
			\item Port C (PC0..PC7) merupakan pin I/O dua arah dan pin fungsi khusus, yaitu TWI, komparator analog dan Timer Oscilator
			\item Port D (PD0..PD7) merupakan pin I/O dua arah dan pin fungsi khusus, yaitu komparator analog, interupsi eksternal dan komunikasi serial
			\item RESET merupakan pin yang digunakan untuk me-reset mikrokontroler
			\item XTAL1 dan XTAL2 merupakan pin masukan clock eksternal
			\item AVCC merupakan pin masukan tegangan untuk ADC
			\item AREF merupakan pin masukan tegangan referensi ADC
		\end{itemize}

\section{ATmega8}

	\subsection{Penjelasan}
		Kali ini saya akan membahas lebih dalam tentang ATMega8. Disini saya fokuskan pada pembahasan tentang fungsi pin, clock, fuse bit, dll.
		Sedikit saya ulas tentang pembahasan saya pada tulisan sebelumnya, bahwa mikrokontroler ATMega8 merupakan mikrokontroler keluarga AVR 8bit. Beberapa tipe mikrokontroler yang “berkeluarga” sama dengan ATMega8 ini antara lain ATMega8535, ATMega16, ATMega32, ATmega328, dll. Yang membedakan antara mikrokontroler yang saya sebutkan tadi antara lain adalah, ukuran memori, banyaknya GPIO (pin input/output), peripherial (USART, timer, counter, dll).Dari segi ukuran fisik, ATMega8 memiliki ukuran fisik lebih kecil dibandingkan dengan beberapa mikrokontroler yang saya sebutkan diatas. Namun untuk segi memori dan periperial lainnya ATMega8 tidak kalah dengan yang lainnya karena ukuran memori dan periperialnya relatif sama dengan ATMega8535, ATMega32, dll, hanya saja jumlah GPIO lebih sedikit dibandingkan mikrokontroler yang saya sebutkan diatas. Untuk pemahaman lebih lanjut akan saya bahas di bawah ini.
	
	\subsection{Fungsi dan Kebutuhan Pin}
		Pinout IC mikrokontroler ATMega8 yang berpackage DIP dapat dilihat di bawah ini.
		Seperti yang kita lihat ATMega8 memiliki 3 buah PORT utama yaitu PORTB, PORTC, dan PORTD dengan total pin input/output sebanyak 23 pin. PORT tersebut dapat difungsikan sebagai input/output digital atau difungsikan sebagai periperial lainnya.
		
	\subsection{Mikrokontroler ATmega8}
		ATMega8 merupakan mikrokontroler keluarga AVR 8 bit yang banyak digunakan oleh pemula seperti saya. Seperti yang terbaca pada datasheet, mikrokontroler yang kita gunakan ini memiliki ukuran flash memori sebesar 8KB, SRAM sebesar 1KB, dan memori EEPROM sebesar 512 Bytes. Bingung apa bedanya dari ketiga memori tersebut. Baik, saya akan jelaskan sedikit tentang perbedaannya.
	\subsection{Jenis dan ukuran memori ATmega8}
		Flash memori merupakan lokasi penyimpanan program yang kita buat. File hex hasil kompilasi program nantinya akan dimasukkan ke mikrokontroler melalui alat yang disebut downloader/programmer. Nah, file hex tersebut nantinya akan disimpan pada sebuah memori yang disebut flash memori. Pada saat melakukan proses pemograman (coding) biasanya kita memerlukan apa yang disebut dengan variabel atau tempat menampung data.
		
\section{ATMega16}
		AVR ATMega16 merupakan seri mikrokontroler CMOS 8-bit buatan Atmel,berbasis arsitektur RISC (Reduced Instruction Set Computer). Hampir semua instruksi dieksekusi dalam satu siklus clock. AVR mempunyai 32 register general-purpose, timer/counter fleksibel dengan mode compare, interrupt internal dan eksternal, serial UART, programmable Watchdog Timer, dan mode power saving, ADC dan PWM internal. AVR juga mempunyai In-System Programmable Flash on-chip yang mengijinkan memori program untuk diprogram ulang dalam sistem menggunakan hubungan serial SPI. ATMega16. ATMega16 mempunyai throughput mendekati 1 MIPS per MHz membuat disainer sistem untuk mengoptimasi konsumsi daya versus kecepatan proses.
	\subsection{Pembagian Kelas ATMega16}
		AVR dapat dikelompokkan menjadi empat kelas, yaitu keluarga Attiny, keluarga AT902xx, keluarga Atmega, dan keluarga AT86RFxx. Pada dasarnya yang membedakan masing-masing kelas adalah memori, peripheral, dan fungsinya. Silahkan buka www.atmel.com untuk informasi lebih lanjut tentang berbagai variasi AVR. Untuk mikrokontroler AVR yang berukuran lebih kecil, silahkan mencoba Atmega8, Attiny2313 dengan ukuran Flash Memory 2KB dengan dua input analog.
	\subsection{Peta Memori ATMega16}
		Memori Program    Arsitektur ATMega16 mempunyai dua memori utama, yaitu memori data dan memori program.  Selain itu, ATMega16 memiliki memori EEPROM untuk menyimpan data.  ATMega16 memiliki 16K byte On-chip In-System Reprogrammable Flash Memory untuk menyimpan program.  Instruksi ATMega16 semuanya memiliki format 16 atau 32 bit, maka memori flash diatur dalam 8K x 16 bit.  Memori flash dibagi kedalam dua bagian, yaitu bagian program boot dan aplikasi. Bootloader adalah program kecil yang bekerja pada saat sistem dimulai yang dapat memasukkan seluruh program aplikasi ke dalam memori prosesor.
	\subsection{Memori Data ATMega16}
		Memori data AVR ATMega16 terbagi menjadi 3 bagian, yaitu 32 register umum, 64 buah register I/O dan 1 Kbyte SRAM internal.  General purpose register menempati alamat data terbawah, yaitu $00 sampai $1F.  Sedangkan memori I/O menempati 64 alamat berikutnya mulai dari $20 hingga $5F.  Memori I/O merupakan register yang khusus digunakan untuk mengatur fungsi terhadap berbagai fitur mikrokontroler seperti kontrol register, timer/counter, fungsi-fungsi I/O, dan sebagainya.  1024 alamat berikutnya mulai dari $60 hingga $45F digunakan untuk SRAM internal.

\section{Mikrokontroler ATMega328}
	\subsection{Penjelasan}
	ATMega328 merupakan mikrokontroler keluarga AVR 8 bit. Beberapa tipe mikrokontroler yang sama dengan ATMega8 ini antara lain ATMega8535, ATMega16, ATMega32, ATmega328, yang membedakan antara mikrokontroler antara lain adalah, ukuran memori, banyaknya GPIO (pin input/output), peripherial (USART, timer, counter, dll). Dari segi ukuran fisik, ATMega328 memiliki ukuran fisik lebih kecil dibandingkan dengan beberapa mikrokontroler diatas. Namun untuk segi memori dan periperial lainnya ATMega328 tidak kalah dengan yang lainnya karena ukuran memori dan periperialnya relatif sama dengan ATMega8535, ATMega32, hanya saja jumlah GPIO lebih sedikit dibandingkan mikrokontroler diatas.

	ATMega328 memiliki 3 buah PORT utama yaitu PORT B, PORT C, dan PORT D dengan total pin input/output sebanyak 23 pin. PORT tersebut dapat difungsikan sebagai input/output digital atau difungsikan sebagai periperal lainnya.
	\begin{enumerate}
	\item Port B 
		Port B merupakan jalur data 8 bit yang dapat difungsikan sebagai input/output. Selain itu PORTB juga dapat memiliki fungsi alternatif seperti di bawah ini.
		\begin{itemize}
			\item ICP1 (PB0), berfungsi sebagai Timer Counter 1 input capture pin. 
			\item OC1A (PB1), OC1B (PB2) dan OC2 (PB3) dapat difungsikan sebagai keluaran PWM (Pulse Width Modulation).
			\item MOSI (PB3), MISO (PB4), SCK (PB5), SS (PB2) merupakan jalur komunikasi SPI.
			\item Selain itu pin ini juga berfungsi sebagai jalur pemograman serial (ISP).
			\item TOSC1 (PB6) dan TOSC2 (PB7) dapat difungsikan sebagai sumber clock external untuk timer.
			\item XTAL1 (PB6) dan XTAL2 (PB7) merupakan sumber clock utama mikrokontroler.
		\end{itemize}

	\end{enumerate}
\section{ATMega128}
	\subsection{penjelasan}
	Mikrokontroler ATmega 128 merupakan mikrokontroler keluarga AVR yang mempunyai kapasitas flash memori 128KB. AVR (Alf and Vegard’s Risc Processor) merupakan seri mikrokontroler CMOS 8-bit buatan Atmel, berbasis arsitektur RISC (Reduced Instruction Set Computer). Dengan mengeksekusi instruksi kuat dalam satu siklus clock tunggal, ATmega128 mencapai throughput mendekati 1 MIPS per MHz yang memungkinkan perancang sistem untuk mengoptimalkan konsumsi daya melawan kecepatan proses
	Fitur Mikrokontroler ATmega128
	Menurut datasheet ATmega128 yang diambil dari sebuah situs resmi Atmel , fitur-fitur pada mikrokontroler ATmega128 antara lain sebagai berikut:
	\begin{itemize}
		\item a. Mikrokontroler AVR 8 bit mempunyai kemampuan tinggi dengan daya rendah. 
		\item b. Arsitektur canggih RISC
				1) 133 intruksi yang kuat. Kebanyakan Single Clock siklus eksekusi 
		\end{itemize}

	