
Dalam artikel ini kami akan membahas cara menghubungkan arduino ke laptop, untuk dapat memprogram Board Arduino, kita perlu software untuk melakukan programming dan menghubungkannya dengan PC. Langkah awalnya kita harus mendownload IDE arduino, kemudian instal aplikasi tersebut. Setelah di instal coba hubungkan arduino ke laptop, tunggu windows untuk melakukan driver instalation. Biasanya terjadi kegagalan.

/section
Instalasi driver untuk Arduino Uno dengan Windows 7
Hubungkan board anda dan tunggu Windows memulai proses instalasi driver. Setelah menginstal proses ini akan gagal, meskipun sudah melakukan yang terbaik.
1.	Klik Start Menu lalu buka Control Panel
2.	Di Control Panel, masuk ke menu System and Security. Kemudian klik sisem. Setelah tampilan system muncul, klik Device Manger.
3.	Di bagian Ports (COM & LPT), TERLIHAT SEBUAH PORT TERBUKA DENGAN NAMA “Arduino Uno (COMxx)
4.	Lalu klik kanan pada port “Arduino Uno (COMxx)” dan pilih opsi “Update driver Software”
5.	Kemudian oilih opsi “Browse my computer for Driver software”
6.	Yang terakhir, masuk dan pilih file driver uno , dengan nama “ArduinoUNO.if”, file tersebut terletak di dalam folder “Drives” pada software Arduino yang telah di-download .
7.	Windows akan meneruskan instalasi driver Arduino Uno sampai selesai.
