
Dalam artikel ini kami akan membahas cara menghubungkan arduino ke laptop, untuk dapat memprogram Board Arduino, kita perlu software untuk melakukan programming dan menghubungkannya dengan PC. Langkah awalnya kita harus mendownload IDE arduino, kemudian instal aplikasi tersebut. Setelah di instal coba hubungkan arduino ke laptop, tunggu windows untuk melakukan driver instalation. Biasanya terjadi kegagalan.

/section
Instalasi driver untuk Arduino Uno dengan Windows 7
Hubungkan board anda dan tunggu Windows memulai proses instalasi driver. Setelah menginstal proses ini akan gagal, meskipun sudah melakukan yang terbaik.
1.	Klik Start Menu lalu buka Control Panel
2.	Di Control Panel, masuk ke menu System and Security. Kemudian klik sisem. Setelah tampilan system muncul, klik Device Manger.
3.	Di bagian Ports (COM & LPT), TERLIHAT SEBUAH PORT TERBUKA DENGAN NAMA “Arduino Uno (COMxx)
4.	Lalu klik kanan pada port “Arduino Uno (COMxx)” dan pilih opsi “Update driver Software”
5.	Kemudian oilih opsi “Browse my computer for Driver software”
6.	Yang terakhir, masuk dan pilih file driver uno , dengan nama “ArduinoUNO.if”, file tersebut terletak di dalam folder “Drives” pada software Arduino yang telah di-download .
7.	Windows akan meneruskan instalasi driver Arduino Uno sampai selesai.
8.  Setelah selesai, restart Laptop dan PC anda dahulu sebelum digunakan.
9.  USB Arduino tetap keadaan tercolok dan biarkan hingga Restart selesai.
10. Setelah di Restart, Buka Aplikasi Arduino IDEnya lalu siapkan Koding untuk USB Arduino-Nya
11. Setelah di Koding, Langsung di Insert kodinganya ke Port USB Arduino.
12. Jika langkah diatas sudah diikuti dengan baik dan benar. Selanjutnya melakukan Tes Arduino masing-masing kelompok
13. Jika sudah dilakukan Tes Arduinonya, dan Jika Sensor Arduino anda tidak merespon atau tidak nyala kembali ke langkah 11
14. Jika sensor berhasil menyala dan berfungsi dengan baik. Berarti menandakan Arduino anda berjalan sesuai fungsinya masing-masing.
15. Perlu diketahui bahwa setiap koding yang dimasukkan pada arduino harus dianalisa baik-baik untuk menghindari Malfunction (Error Mendadak)
16. Apabila anda sudah mengikuti langkah diatas dengan baik dan benar maka Anda sudah bisa memahami prosedur langkah-langkah ini.
