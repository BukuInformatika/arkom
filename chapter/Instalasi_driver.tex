% Nama Kelompok : 
% Kelas : D4 TI 1A
% Anggota : 
% 1. Harun    1174027
% 2. Fahmi    1174021
% 3. Kukuh    1174016
% 4. Izzah    1174013
% 5. Rizal    1174014
% 6. Lawimner 1174030

Artikel ini berisi mengenai instalasi driver
\section{Pengertian}
Pengertian driver komputer adalah sekumpulan perangkat lunak untuk memperkenalkan perangkat keras kepada perangkat lunak sistem operasi, dengan perangkat lunak ini sistem mampu untuk menggunakan perangkat keras dengan baik, untuk performace hardware yang lebih baik ada kalanya kamu mengupdate driver tersebut setiap minggu, hal ini juga termasuk meminimalisir terjangkit virus pada driver.

Ibarat kayak  namanya, driver dapat kita ibaratkan seperti sopir yang siap mengantarkan kita kemana saja [karena kita tidak bisa nyetir].
Jadi sebelum ada sopir, selama itu juga mobil tidak akan bisa berjalan dan kita tidak akan pernah sampai ketujuan.
Contohnya  : Saat kita ingin memutar file mp3 tetapi belum menginstall driver.
Karena disaat perangkat audio tidak bisa terbaca oleh windows, maka windows akan menganggap laptop tersebut tidak mempunyai perangkat audio jadi laptop tidak akan mengeluarkan suara dan tentunya file mp3 tidak bisa diputar.

\section{Fungsi driver komputer}
Fungsi driver computer seperti pengertianya yang mempunyai fungsi penerjemah segala instruksi yang dilakukan melalui perantara system operasi
Fungsinya adalah untuk menyediakan transparasi yang berfungsi untuk penerjemah antara hardware dengan system operasi 
Dengan itu Fungsi driver komputer penting sekali , saat anda mendownload driver jangan asal asalan
Harus sesuai hardware dan sistem operasi yang digunakan.

\section {Berapa driver yang harus diinstall}

Sebuah komputer tersusun atas banyak hardware.  kita tidak perlu untuk menginstall drivernya  semuanya
sudah bisa dihandle oleh driver bawaan yang ada pada sistem operasi.
Namun secara umum setelah melakukan install ulang, setidaknya ada 7 hardware yang drivernya harus terinstall, hardware tersebut adalah : VGA, Lan, Wifi, Chipset, Audio, Touchpad [pada laptop], Bluetooth [ jika ada].
Fungsi dari driver tersebut tentunya agar masing-masing hardware yang saya sebutkan tadi berfungsi sebagaimana mestinya.
jika ingin internetan menggunakan jaringan Wifi kantor, tetapi anda tidak menginstall driver untuk wifi, maka otomatis anda tidak akan bisa menangkap sinyal Wifi karena hardware Wifi pada laptop belum aktif.

\section {Cara Mendapatkan Driver}
Ada beberapa cara ataupun teknik yang kami rekomendasikan untuk mendapatkan driver komputer yang sesuai dengan spesifikasi computer yang dimiliki , yaitu :
1.	Dengan cara menggunakan CD 
Driver yang berasal dari CD driver bawaan adalah cara ataupun teknik penggunaan driver yang paling tepat , cepat dan aman untuk mendapatkan driver yang kompatibel.
Hal ini dikarenakan cd driver tersebut telah sesuai dengan hardware yang digunakan oleh perangkat komputer anda. 
Mungkin ada beberapa driver yang verlu d update jika lewat cd karena mungkin terlalu jadul.
Terlebih lagi jika anda yang menggunakan laptop, setelah selesai melakukan instalasi ulang, anda diharuskan langsung menginstall driver yang sesuai dengan leptop, sehingga laptop dapat digunakan untuk segala kegiatan komputasi.

Berbeda dengan laptop, komputer desktop rakitan agak sedikit lebih ribet ketika menginstall driver.

Jika komputer rakitan yang anda gunakan memiliki banyak hardware tambahan seperti vga card addons, kartu jaringan wifi, bluetooth, dll, anda harus menginstall driver untuk hardware tambahan tersebut secara terpisah.

Tetapi tidak perlu pusing, karena masing-masing hardware tersebut selalu disertai dengan cd driver yang sudah pasti kompatibel.

Penting untuk diketahui !‼
Disaat kondisi tertentu  , keberadaan CD Driver tidak dapat digunakan atau difungsikan untuk menginstall driver . Mengapa hal itu terjadi ? Hal ini bisa terjadi disaat kita melakukan upgrade system operasi yang tidak disupport oleh pihak manufaktur . Contohnya , kita memiliki laptop , dimana keberadaan awalnya didesain untuk system operasi windows xp , maka disaat itulah kita menggunakan system operasi terbaru seperti system windows 7.


