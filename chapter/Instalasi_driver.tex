% Nama Kelompok : 
% Kelas : D4 TI 1A
% Anggota : 
% 1. Harun    1174027
% 2. Fahmi    1174021
% 3. Kukuh    1174016
% 4. Izzah    1174013
% 5. Rizal    1174014
% 6. Lawimner 1174030

Artikel ini berisi mengenai instalasi driver
\section{Pengertian}
Pengertian driver komputer adalah sekumpulan perangkat lunak untuk memperkenalkan perangkat keras kepada perangkat lunak sistem operasi, dengan perangkat lunak ini sistem mampu untuk menggunakan perangkat keras dengan baik, untuk performace hardware yang lebih baik ada kalanya kamu mengupdate driver tersebut setiap minggu, hal ini juga termasuk meminimalisir terjangkit virus pada driver.

Ibarat kayak  namanya, driver dapat kita ibaratkan seperti sopir yang siap mengantarkan kita kemana saja [karena kita tidak bisa nyetir].
Jadi sebelum ada sopir, selama itu juga mobil tidak akan bisa berjalan dan kita tidak akan pernah sampai ketujuan.
Contohnya  : Saat kita ingin memutar file mp3 tetapi belum menginstall driver.
Karena disaat perangkat audio tidak bisa terbaca oleh windows, maka windows akan menganggap laptop tersebut tidak mempunyai perangkat audio jadi laptop tidak akan mengeluarkan suara dan tentunya file mp3 tidak bisa diputar.
