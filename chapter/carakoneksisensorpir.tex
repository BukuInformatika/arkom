%Kelompok BSD
%Arjun Yuda Firwanda
%Dwi Septiani Tsaniyah
%Dwi Yulianingsih
%Ervanda Rambu Anarky
%Jeremia Wahyudi Sianturi

\section {Cara Mengkoneksi Sensor Pir}

Sensor Pir adalah sensor yang memiliki infrared yang memancar pada sensor yang mendeteksi adanya gerakan tangan.
Dengan demikian sensor pir bekerja pada sebuah gerakan dan infrared akan menangkap sinar dari sebuah gerakan yang akan mendeteksi si sensor pirnya dan dengan codingan yang benar angka akan mucul pada serial monitor.
Dalam melakukan percobaan sebuah sensor pir hendaknya mengetahui terlebih dahulu apa saja yang diperlukan dan bagaimana codingan yang benar untuk mengetahui sebuah sensor itu bergerak atau terdeteksi. 

\subsection {Cara Merakit Sensor Pir dan Arduino}

Sebelum merakit sensor pir dan arduiono berikut yang perlu dilakukan sebelum merakit.
1. Pastikan kita mengetahui komponen sensor pir.
2. Mengetahui fungsi dan cara kerja sensor pir.

\subssubsection {Tutorial Merakit Sensor Pir}
Alat yang diperlukan:
1. Arduino Uno.
2. Sensor Pir.
3. Lampu Led (warna bebas).
4. Kabel Jumper Male to Female (3 buah warna).
5. Kabel USB.
6. PC.

\subsection {Kegunaan sensor PIR}
Sensor ini bekerja dengan cara membaca gerakan pada jarak tertentu, jarak ini yang sangat mempengaruhi sensor semakin dekat jarak benda bergerak maka semakin gampang pula sensor membaca gerakan.
Biasanya sensor ini digunakan pada pintu mall otomatis.