\section Output Device Arduino 
\subsection{LED}
LED adalah lampu kecil (singkatan dari \"light emitting diode\") yang bekerja dengan daya yang relatif kecil. Dewan Arduino memiliki satu built-in pada pin digital 13.
Untuk mengedipkan LED hanya membutuhkan beberapa baris kode. Hal pertama yang kita lakukan adalah mendefinisikan sebuah variabel yang akan menahan jumlah pin yang 
terhubung dengan LED. Kita tidak perlu melakukan ini (kita bisa menggunakan nomor pin di seluruh kode) tapi itu membuat lebih mudah untuk mengganti pin yang berbeda. 
Kami menggunakan variabel integer (disebut int). Seperti lampu pijar dan tidak seperti kebanyakan lampu neon (misalnya tabung dan lampu neon kompak atau CFL), LED mencapai kecerahan penuh tanpa memerlukan waktu pemanasan kehidupan pencahayaan neon juga dikurangi dengan sering menyalakan dan mematikan. Biaya awal LED biasanya lebih tinggi. Degradasi pewarna LED dan bahan kemasan mengurangi keluaran cahaya sampai batas tertentu dari waktu ke waktu.
Beberapa lampu LED dibuat untuk menjadi pengganti drop-in yang kompatibel secara langsung untuk lampu pijar atau lampu neon. Kemasan lampu LED dapat menunjukkan output lumen, konsumsi daya dalam watt, suhu warna pada kelvin atau deskripsi, kisaran suhu operasi, dan kadang-kadang watt setara lampu pijar dari keluaran bercahaya serupa.
\subsection{Resistor}
Sebuah resistor adalah komponen listrik dua terminal pasif yang menerapkan hambatan listrik sebagai elemen rangkaian. Di sirkuit elektronik, resistor digunakan 
untuk mengurangi arus, menyesuaikan level sinyal, membagi tegangan, elemen aktif biasa, dan menghentikan jalur transmisi, di antara kegunaan lainnya. Resistor 
berdaya tinggi yang dapat mengusir banyak daya listrik karena panas dapat digunakan sebagai bagian kontrol motor, dalam sistem distribusi tenaga, atau sebagai 
beban uji untuk generator. Resistor tetap memiliki tahanan yang hanya sedikit berubah dengan suhu, waktu atau voltase operasi. Resistor variabel dapat digunakan 
untuk mengatur elemen rangkaian (seperti kontrol volume atau lampu dimmer), atau sebagai alat penginderaan untuk panas, cahaya, kelembaban, gaya, atau aktivitas 
kimia. Resistor adalah elemen umum jaringan listrik dan sirkuit elektronik dan ada di mana-mana di peralatan elektronik. Resistor praktis sebagai komponen diskrit dapat terdiri dari berbagai senyawa dan bentuk. Resistor juga diimplementasikan dalam sirkuit terpadu.
Fungsi kelistrikan resistor ditentukan oleh resistannya: resistor komersial yang umum dibuat dengan kisaran lebih dari sembilan orde. Nilai nominal resistansi berada di dalam toleransi manufaktur, yang ditunjukkan pada komponen.

\subsection(Buzzer)
Bel atau pager adalah perangkat sinyal audio, yang mungkin mekanis, elektromekanis, atau piezoelektrik (piezo singkatnya). Khas penggunaan buzzer dan beepers termasuk 
perangkat alarm, timer, dan konfirmasi masukan pengguna seperti klik mouse atau keystroke.

\subsubsection(Sejarah)
\begin{itemize}
\item Elektromekanis
Bels listrik ditemukan pada tahun 1831 oleh Joseph Henry. Mereka terutama digunakan di bel pintu awal sampai mereka berhenti di awal tahun 1930an untuk mendukung
lonceng musik, yang memiliki nada lebih lembut.
\item Piezoelektrik
Cahaya piezoelektrik, atau buzz piezo, seperti yang kadang-kadang disebut, ditemukan oleh pabrikan Jepang dan dilengkapi dengan beragam produk selama tahun 1970an 
sampai 1980an. Kemajuan ini terutama terjadi karena usaha koperasi oleh perusahaan manufaktur Jepang. Pada tahun 1951, mereka mendirikan Barium Titanate Aplikasi 
Research Committee, yang memungkinkan perusahaan untuk menjadi \"kompetitif koperasi\" dan membawa beberapa inovasi piezoelektrik dan penemuan.
\end(itemize)

\subsubsection(Jenis-jenis Buzzer)
\begin(itemize)
\item Elektromekanis
Perangkat awal didasarkan pada sistem elektromekanis yang identik dengan bel listrik tanpa gong logam. Demikian pula, relay dapat dihubungkan untuk mengganggu 
arus penggeraknya sendiri, menyebabkan kontak buzz. Seringkali unit ini berlabuh ke dinding atau plafon untuk menggunakannya sebagai papan suara. Kata \"bel\" 
berasal dari suara serak yang dibuat oleh buzz elektromekanis.
\item Mekanis
Joy buzzer adalah contoh bel yang mekanis dan mereka memerlukan driver. Contoh lain dari mereka adalah bel pintu.
