\begin{figure}[ht]
\centerline{\includegraphics[width=0.25\textwidth]{/figures/hexadec.png}}
\caption{hexadecimal}
\label{hexadec}
\end{figure}

\section{pengertian hexadecimal}
	Hexadecimal adalah sebuah sistem bilangan yang menggunakan sebuah simbol.Dalam hexadecimal Terdapat beberapa simbol yang bisa digunakan di sistem bilangan ini.Berbeda dengan bilangan decimal.hexadecimal menggunakan angka 0 sampai 1, di bilangan hexadecimal ini tidak menggunakan angka semua melainkan ada beberapa simbol yang menggunakan huruf.jumlah simbol yang yang berasal dari angka 1 sampai 9 berjumlah 16 simbol, ditambah dengan 6 simbol lainnya yang menggunakan huruf dari A sampai F.Hexadecimal bisa digunakan untuk menampilkan nilai alamat memori dan pemrograman komputer.Teknik penjumlahan dan pengurangan pada bilangan hexadecimal hampir sama dengan penjumlahan dan pengurangan pada bilangan biner,octal dan decimal, tetapi jika terjadi carry 1 atau borrow 1, maka angka 1 tersebut bernilai 16. Carry akan terjadi apabila penjumlahan lebih dari 15 misalnya 8+8=10. Sedangkan borrow terjadi apabila angka yang dikurangi lebih kecil dari pengurang, misalnya 45-6=. \ref{hexadec}
	\cite {schwarz1997implementation}
\section{operasi penjumlahan pada bilangan hexadesimal}
penjumlahan bilangan hexadesimal dapat dilakukan secara sama dengan penjumlahan bilangan octal, dengan langkah-langkah sebagai berikut: \begin{enumerate}
			\item tambahkan masing-masing kolom secara desimal. 
			\item rubah dari hasil desimal ke hexadesimal .
			\item tuliskan hasil dari digit paling kanan dari hasil hexadesimal. 
			\item jika hasil penjumlahan setiap kolom terdiri dari dua digit, maka digit paling kiri merupakan carry of untuk penjumlahan pada kolom selanjutnya.
			\end{enumerate}
\section{operasi pengurangan pada bilangan hexadesimal}
pengurangan mudah diselesaikan jika dikerjakan dengan rapi yaitu memperhatikan lajur-lajur perseratusan, persepuluhan, satuan, puluhan, ratusan, dan sebagainya. untuk menyelesaikan pengurangan bilangan hexadesimal, ikuti langkah-langkah ini :
	\begin{enumerate}
		\item tulis kedua bilangan bersusun ke bawah, sejajarkan sehingga koma hexadesimal membentuk baris lurus.
		\item tambahkan nol agar bilangan memiliki panjang yang sama.
		\item kemudian kurangkan, jangan lupa mencantumkan koma hexadesimal pada jawabannya.
	\end{enumerate}
\section{operasi perkalian pada bilanagn decimal}
cara mengoperasikan perkalian bilangan hexadecimal sama seperti opersi perkalian pada bilanagan decimal. Caranya sebagai berikut :
	\begin{enumerate}
		\item Kalikan masing-masing kolom secara decimal
		\item Kemudian ubah dari hasil decimal ke hexadecimal
		\item Tuliskan hasil dari digit paling kanan dari hasil hexadecimal
		\item Jika hasil perkalian tiap-tiap kolom terdiri dari dua digit, maka digit yang berada pada posisi yang paling kiri merupakan carry of untuk ditambahkan pada hasil perkalian kolom berikutnya
	\end{enumerate}
\section{operasi aritmatika pada bilangan hexadesimal}
penjumlahan bilangan hexadesimal dapat dilakukan secara sama dengan penjumlahan bilangan octal, dengan langkah-langkah sebagai berikut: 1) tambahkan masing-masing kolom secara desimal, 2) rubah dari hasil desimal ke hexadesimal, 3) tuliskan hasil dari digit paling kanan dari hasil hexadesimal, 4) jika hasil penjumlahan setiap kolom terdiri dari dua digit, maka digit paling kiri merupakan carry of untuk penjumlahan pada kolom selanjutnya.


\section {operasi pembagian pada hexadecimal}
Pembagian pada bilangan Hexadecimal sama seperti pembagian pada bilangan decimal. adapun langkah pembagian pecahan decimal dengan cara mudah, yaitu :
	Contoh : 
	\begin{verbatim}
	0,625 : 0,25
	\end{verbatim}
	\begin{enumerate}
	\item Reken Koma ( tidak menghiraukan tanda koma), sehingga bilangan 0,625 menjadi 625 dan 0,25 menjadi 25.
	\item Hitung banyaknya angka di belakang koma bilangan yang dibagi dengan banyaknya angka di belakang bilangan pembagi, jika hasil pengurangannya adalah positif 1,2,3 dst, maka hasil pembagian pada point 2 masih harus dibagi 10, 100, 1000 dst atau masih harus ada 1,2,3 angka di belakang koma, dan jika hasil pengurangan adalah 0 maka hasil pembagian pada point 2 tetap. lalu jika hasil pengurangannya adalah negatif (-1,-2,-3 dst, maka hasil pembagian pada point masih harus dikalikan dengan 10, 100, 1000, dst). Terdapat 3 angka di belakang koma pada bilangan yang dibagi dan terdapat 2 angka pada bilangan pembagi. 
	\end{enumerate}

\section{operasi pembagian bilangan decimal}
Untuk jumlah Desimal pada jawaban, kita tinggal mengurangkankan jumlah Desimal pada angka mengalikan 30 x 75 dua desimal, 
dengan angka yang dikalikan 12 x 3 satu desimal, 
dua desimal dikurangi satu desimal =satu desimal. berarti satu angka dibelakang koma satu Desimal, yaitu 5, satu angka dihitung dari belakang makanya pada jawaban tertulis 2,5
 Contoh Pembagian Pecahan Desimal yang lain misal : 
 30,75 : 1,23 = 25
jumlah Desimal pada jawaban, kita tinggal mengurangkankan jumlah Desimal pada angka mengalikan 30 x 75 dua desimal, 
dengan angka yang dikalikan 1 x 23 dua desimal, 
dua desimal dikurangi dua desimal = habis.berarti tidak ada angka dibelakang koma, makanya pada jawaban tertulis 25
 Contoh Pembagian Pecahan Desimal yang lain lagi misal : 
 307,5 : 1,23 = 250
\section{Konversi Bilangan hexadecimal}
Berikutnya kami akan menunjukkan cara-cara mengkonversikan bilangan hexadecimal ke bilangan yang lain seperti ke bilangan decimal dan ke bilangan biner
\section{Bilangan hexadecimal ke bilangan decimal}
Pertama-tama kalikan setiap digit bilangan decimal dengan nilai bilangan hexadecimal yaitu 16. Setelah itu pangkatkan nilai hexadecimal dari 0 yang dimulai dari sebelah paling kanan,setelah itu jumlahkan angka dari hasil perkalian tersebut.
contoh : 
Ubahlah \[ 7C6_16 \] menjadi decimal
		\[ 7 x 16^2 + C x 16^1 + 6 x 16^0\]
		\[ 1792 + 192 + 6 = 1990_10 \]

\section{bilangan hexadecimal ke oktal}
Caranya, yaitu dengan menerjemahkan angka hexadecimal de dalam bilangan biner melalui tabel, lalu diterjemahkan lagi menjadi ke bentuk oktal dengan cara mengambil 3 karaker di kanan, lalu setelah itu cocokkan dengan angka yang ada di tabel. Jika angka terakhirnya tidak cukup atau kurang dari 3 karakter, maka bisa ditambahkan angka 0 di sebelah kiri angka agar memudahkan pengoperasian

\section{Contoh Cara Mengkonversi Bilangan Hexadecimal menjadi Biner}
 Untuk Hal ini Mengkonversi Bilangan Hexadecimal akan terlihat mudah apabila kalian bisa memahami untuk mengubahnya jadi angka unik, Sebagai contoh :
 F12 = 1111 0001 0010
 F mempunyai nilai 1111 atau F setara dengan Angka 15
 1 mempunyai nilai 0001 pada tabel Biner
 2 mempunyai nilai 0010 pada tabel biner
 
 Jika dipahami, Mungkin anda berpikir mengapa ada perubahan nilai angka 1 dan 2. Karena ini sama halnya dengan menghitung 4 digit dari belakang.
 Misalkan 0 = 0000, 1 = 0001 ,dan 2 = 0010. Apabila diperhatikan terdapat angka 1 dibelakang biner setiap tingkatan. Penerapan angka biner ini
 akan sangat mudah jika kalian bisa memahami angka yang akan dikonversikan dan memahami perhitungan logika yang akurat.

\section {Kegunaan Hexadecimal}
 Hexadecimal biasanya digunakan untuk:
 \begin{itemize}
   \item Spasi pada URL suatu website
   \item Kode warna dalam html, html merupakan bahasa pemrograman yang sering digunakan dalam membangun suatu website. kode warna ini dinamakan HEX HTML, yang biasanya itu digunakan untuk mengganti DECIMAL RGB dikarenakan penggunaannya yang lebih mudah. Kode ini sering juga digunakan untuk software grafik, contohnya itu seperti Photoshop, Gimp, dan sebagainya.
   \item Pada bahasa pemrograman C++
   \item Hitungan dengan basis 16
 \end{itemize}

\section{kesimpulan}
 hexadecimal adalah sebuah bilangan yang menggunakan simbol dari angka 0 sampai 9, ditambah dengan 6 simbol lainnya yang menggunakan huruf abjad dari huruf A sampai F.Sistem bilangan ini digunakan untuk menampilkan nilai alamat memori dalam pemrograman komputer.Di bilangan Hexadecimal juga terdapat penjumlahan,pengurangan,perkalian dan pembagian.Bilangan hexa decimal juga dapat di konversi ke bilangan lain,misalnya dari hexadecimal ke oktal.