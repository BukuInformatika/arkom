\section{Cara menghubungkan arduino ke laptop}
Dalam artikel ini kami akan membahas cara menghubungkan arduino ke laptop, untuk dapat memprogram Board Arduino, kita perlu software untuk melakukan programming dan menghubungkannya dengan PC. Langkah awalnya kita harus mendownload IDE arduino, kemudian instal aplikasi tersebut. Setelah di instal coba hubungkan arduino ke laptop, tunggu windows untuk melakukan driver instalation. Biasanya terjadi kegagalan.
Setelah arduino terhubung kita lanjutkan ke Langkah selanjutnya hubungkan arduino dengan PC menggunakan kabel usb yang tadi hingga PC mendeteksi arduino tersebut. Kemudian buka control panel windows,lalu buka device manager. Setelah itu cari ports (COM dan LPT), disana ada port port arduino UNO (COMxx), click kanan dan pilih "update driver software" option.
Lalu pilih "Browse my computer for driver software" option. Cari driver file dengan nama "arduino.inf", di folder drivers. Folder dapat ditemukan ditempat menginstal software IDE Arduino.
Setelah itu, dengan sendirina windows akan menyelesaikan instalasi driver. Kemudian sketch akan terbuka pada windows software Arduino anda,klik tipe board Arduino anda, karena anda menggunakan arduino uno, maka klik Arduino Uno.
Setelah memilih board yang sesuai dengan perangkat yang terhubung, maka selanjutnya anda harus memilih port yang menghubungkan PC dengan Arduino.
<<<<<<< HEAD
\section
Berikut adalah cara menghubungkan arduino dengan laptop, sebelum dihubungkan kekomputerdownload arduino software terlebih dahulu.
1)Hubungkan kabel arduino dengan laptop
2)Buka device manager dengan cara klik kanan mycomputer
3)Klik other device jika arduino tidak muncul klik kanan dan update driver software kemudian pilih Browse my komputer for driver software kemudian pilih tempat arduino software disimpan.
4)Jika update software berhasil, klik tools di software arduino untuk pilih jenis board dan port yang dihubungkan dengan komputer.
=======
setelah program tersebut dapat terupload, di notifikasi akan muncul tanda atau pemberitahuan bahwa arduino sudah terhubung dengan laptop. Salah satu ciri arduino terhubung dengan laptop adalah LED 13 dari arduino akan berkedip-kedip selama 1 detik disetiap kedipannya.
setelah itu anda tinggal memasukkan kodingan agar arduino bekerja sesuai dengan kehendak anda.
Jika Prosedur diatas diikuti dengan baik dan benar maka arduino berjalan sesuai perintahnya.
Apabila terjadi masalah mengenai error menjalankan arduinonya berarti ada beberapa bug yang membuat Error ini terjadi.
Saat melakukan koding di Arduino pastikan tidak ada miss type atau typo dalam mengoding. Kesalahan sedikit akan membuat driver rusak dari kodingan itu sendiri
>>>>>>> 6a644cdcada4ca9e8148c5e64ee6e9fc03aacaab
\section
Berikut adalah cara menghubungkan arduino dengan laptop, sebelum dihubungkan kekomputerdownload arduino software terlebih dahulu.
1)Hubungkan kabel arduino dengan laptop
2)Buka device manager dengan cara klik kanan mycomputer
3)Klik other device jika arduino tidak muncul klik kanan dan update driver software kemudian pilih Browse my komputer for driver software kemudian pilih tempat arduino software disimpan.
4)Jika update software berhasil, klik tools di software arduino untuk pilih jenis board dan port yang dihubungkan dengan komputer.
