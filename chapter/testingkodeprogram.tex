%Kelompok BSD
%Arjun Yuda Firwanda
%Dwi Septiani Tsaniyah
%Dwi Yulianingsih
%Ervanda Rambu Anarky
%Jeremia Wahyudi Sianturi

Testing Kode Program Sensor Gerak

\section {Testing Program}
Sistem Testing (pengujian)
Pengujian perangkat lunak (bahasa Inggris: software testing) merupakan suatu investigasi yang dilakukan untuk mendapatkan informasi mengenai kualitas dari produk atau layanan yang sedang diuji (under test).Pengujian perangkat lunak juga memberikan pandangan mengenai perangkat lunak secara obyektif dan independen, yang bermanfaat dalam operasional bisnis untuk memahami tingkat risiko pada implementasinya. Teknik-teknik pengujian mencakup, namun tidak terbatas pada, proses mengeksekusi suatu bagian program atau keseluruhan aplikasi dengan tujuan untuk menemukan bug perangkat lunak (kesalahan atau cacat lainnya).
Cara mengakses sensor PIR menggunakan Arduino

Sensor PIR (Passive Infra Red) atau disebut dengan Sensor Gerak merupakan sensor yang digunakan untuk mendeteksi adanya benda atau sebuah gerakan tangan untuk mentransfer dengan cara infra red atau sinar merah yang berasal dari gerakan tangan. Tidak hanya dengan pendeteksian pancaran sinar infra merah melaikan sebuah infra red.
Komponen elektronika ini mempunyai sifat pasif, yang artinya tidak dapat memancarkan sinar infra merah secara independen tetapi hanya menerima radiasi sinar infra merah dari luar.