%Kelompok BSD
%Arjun Yuda Firwanda
%Dwi Septiani Tsaniyah
%Dwi Yulianingsih
%Ervanda Rambu Anarky
%Jeremia Wahyudi Sianturi

Testing Kode Program Sensor Gerak

\section {Testing Program}
Sistem Testing (pengujian)
Pengujian perangkat lunak (bahasa Inggris: software testing) merupakan suatu investigasi yang dilakukan untuk mendapatkan informasi mengenai kualitas dari produk atau layanan yang sedang diuji (under test).Pengujian perangkat lunak juga memberikan pandangan mengenai perangkat lunak secara obyektif dan independen, yang bermanfaat dalam operasional bisnis untuk memahami tingkat risiko pada implementasinya. Teknik-teknik pengujian mencakup, namun tidak terbatas pada, proses mengeksekusi suatu bagian program atau keseluruhan aplikasi dengan tujuan untuk menemukan bug perangkat lunak (kesalahan atau cacat lainnya).
Dalam dunia software perubahan requiretment adalah hal yang sangat biasa dan sulit ditebak maupun dihindari, dapat dipastikan karena adanya kesalahan-kesalahan pada software maupun pengguna dalam menganalisis objek yang ada. Sebuah metodelogi dalam pengembangan perangkat lunak yang dirancang untuk kondisi yang serba dinamis dan tidak terprediksi seperti masalah perubahan requiretment pada intinya akan kembali pada cara agar pengembangan sofware testing kode dapat berjalan dengan baik dan memberi feedback yang sesuai agar kesalahan yang ada dapat di minimalisir.

Process software testing merupakan process yang sangat penting dalam dunia perangkat lunak. karena dengan kita menerapkan process ini di dalam alur pengembangan software kita, maka dengan ini kita dapat menjamin kualitas dari software yang kita buat (setidaknya dalam hal pemenuhan functional requirement). Lalu, apa saja process software testing yang harus kita lakukan? software developer dan baca-baca buku tentang software development, setidaknya ada 3 jenis testing yang dapat kita lakukan yaitu: Unit Testing, Integration Testing dan User Acceptance Testing (UAT)

Unit Testing
Unit testing adalah sebuah percobaan dalam sebuah project. Seorang programmer harus melakukan banyak testing sehingga apa-apa saja yang kurang tersebut harus mengetahuinya dan dapat mengevaluasi kebenaran datanya. Testing harus dilakukan secara bertahap dan dilakukan sebanyak mungkin untuk sebuah project. Programmer yang cerdas seharusnya dapat melakukan testing menurut teori dan diterapkan dalam sebuah praktik.

Satu hal yang harus diingat dalam melakukan unit testing tersebut adalah jika unit testing tersebut adalah testing yang bersifat independen dan isolated. Sebuah method / fungsi dapat dikatakan sebagai independen jika fungsi tersebut tidak bergantung dengan hasil dari fungsi yang lain sedangkan yang dimaksud dengan isolated adalah bahwa fungsi yang di test tidak boleh melakukan akses ke “luar” seperti misalnya mengakses database, file ataupun membutuhkan koneksi jaringan.

Cara mengakses sensor PIR menggunakan Arduino

Sensor PIR (Passive Infra Red) atau disebut dengan Sensor Gerak merupakan sensor yang digunakan untuk mendeteksi adanya benda atau sebuah gerakan tangan untuk mentransfer dengan cara infra red atau sinar merah yang berasal dari gerakan tangan. Tidak hanya dengan pendeteksian pancaran sinar infra merah melaikan sebuah infra red.
Komponen elektronika ini mempunyai sifat pasif, yang artinya tidak dapat memancarkan sinar infra merah secara independen tetapi hanya menerima radiasi sinar infra merah dari luar.

Kegunaan dari sensor ini biasanya digunakan dalam perancangan detektor pergerakan. Dikarenakan semua benda yang memancarkan energi radiasi, akan terdeteksi oleh sensor ini pada saat infra merah dari sensor PIR mendeteksi dengan perbedaan suhu tertentu.

Contoh dalam kehidupan sehari – hari yaitu pada saat memasuki pintu Mall yang membuka dengan otomatis saat kita akan memasuki area dalam Mall.
<br>
Cara kerja pembacaan pada sensor PIR

pantulan dari infra merah yang telah masuk melalui lensa fresnel dan mengenai sensor akan menimbulkan energi panas dari energi panas tersebut maka sensor akan mengeluarkan arus listrik.
Sensor pyroelektrik tersebut disasari oleh beberapa bahan yang didalamnya mengandung galium nitrida (GaN), cesium nitrat (CsNo3) serta litium tantalate (LiTaO3).
Arus listriktersebut yang akan memunculkan tegangan analog yang kemudian dikenali oleh sensor. setelah itu sinyal akan dikuatkan dan dibandingkan oleh komparator dengan tegangan masing-masing (hasil yang diberikan berupa sinyal 1-bit).

Jadi sensor PIR hanya akan mengeluarkan logika 0 dan 1 saja. Jika logika 0, kondisi saat sensor tidak mendeteksi adanya pancaran infra merah dan sedangkan pada saat kondisi logika 1 kondisi saat sensor mendeteksi infra merah.

Sensor PIR atau disebut dengan sensor gerak didesain dan dirancang sedemikian rupa untuk mendeteksi pancaran sinar infra merah dengan panjang gelombang 8 sampai14 mikrometer, diluar ukuran tersebut gelombang infra red sensor tidak akan mendeteksinya atau tidak dapat terbaca oleh sensor tersebut. Pendeteksian sensor PIR  dapat dilakukan dengan gerakan tangan, dengan cara menggerakkan tangan ke arah sensor PIR.
Pada kalangan manusia yang memiliki suhu badan, suhu tersebut adalah penyebab dimana manusia bisa menghasilkan sinar infra merah dengan panjang gelombang berkisar 9-10 mikrometer (nilai standar yang digunakan 9,4 mikrometer), dengan panjang gelombang tersebut maka akan terbaca oleh sensor PIR. Pada umumnya sensor tersebut memang dirancang agar dapat mendeteksi gerakan manusia yang kemudian bisa membuat sensor tersebut menyala atau berfungsi.

Pada saat sensor itu mulai terpasang dan sensor tidak bisa mendeteksi karena adanya benda yang bergerak di depannya maka lampu LED tersebut secara default akan langsung padam, dan sensor akan menyala lagi dalam waktu yang sangat delay yang telah diatur sedemikian rupa pada potensiometer sensor PIR (Passive Infra Red) .