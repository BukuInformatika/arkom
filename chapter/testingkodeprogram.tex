%Kelompok BSD
%Arjun Yuda Firwanda
%Dwi Septiani Tsaniyah
%Dwi Yulianingsih
%Ervanda Rambu Anarky
%Jeremia Wahyudi Sianturi

Testing Kode Program Sensor Gerak

\section {Testing Program}
Sistem Testing (pengujian)
Pengujian perangkat lunak (bahasa Inggris: software testing) merupakan suatu investigasi yang dilakukan untuk mendapatkan informasi mengenai kualitas dari produk atau layanan yang sedang diuji (under test).Pengujian perangkat lunak juga memberikan pandangan mengenai perangkat lunak secara obyektif dan independen, yang bermanfaat dalam operasional bisnis untuk memahami tingkat risiko pada implementasinya. Teknik-teknik pengujian mencakup, namun tidak terbatas pada, proses mengeksekusi suatu bagian program atau keseluruhan aplikasi dengan tujuan untuk menemukan bug perangkat lunak (kesalahan atau cacat lainnya).
Cara mengakses sensor PIR menggunakan Arduino

Sensor PIR (Passive Infra Red) atau disebut dengan Sensor Gerak merupakan sensor yang digunakan untuk mendeteksi adanya benda atau sebuah gerakan tangan untuk mentransfer dengan cara infra red atau sinar merah yang berasal dari gerakan tangan. Tidak hanya dengan pendeteksian pancaran sinar infra merah melaikan sebuah infra red.
Komponen elektronika ini mempunyai sifat pasif, yang artinya tidak dapat memancarkan sinar infra merah secara independen tetapi hanya menerima radiasi sinar infra merah dari luar.

Kegunaan dari sensor ini biasanya digunakan dalam perancangan detektor pergerakan. Dikarenakan semua benda yang memancarkan energi radiasi, akan terdeteksi oleh sensor ini pada saat infra merah dari sensor PIR mendeteksi dengan perbedaan suhu tertentu.

Contoh dalam kehidupan sehari – hari yaitu pada saat memasuki pintu Mall yang membuka dengan otomatis saat kita akan memasuki area dalam Mall.
<br>
Cara kerja pembacaan pada sensor PIR

pantulan dari infra merah yang telah masuk melalui lensa fresnel dan mengenai sensor akan menimbulkan energi panas dari energi panas tersebut maka sensor akan mengeluarkan arus listrik.
Sensor pyroelektrik tersebut disasari oleh beberapa bahan yang didalamnya mengandung galium nitrida (GaN), cesium nitrat (CsNo3) serta litium tantalate (LiTaO3).
Arus listriktersebut yang akan memunculkan tegangan analog yang kemudian dikenali oleh sensor. setelah itu sinyal akan dikuatkan dan dibandingkan oleh komparator dengan tegangan masing-masing (hasil yang diberikan berupa sinyal 1-bit).

Jadi sensor PIR hanya akan mengeluarkan logika 0 dan 1 saja. Jika logika 0, kondisi saat sensor tidak mendeteksi adanya pancaran infra merah dan sedangkan pada saat kondisi logika 1 kondisi saat sensor mendeteksi infra merah.

Sensor PIR atau disebut dengan sensor gerak didesain dan dirancang sedemikian rupa untuk mendeteksi pancaran sinar infra merah dengan panjang gelombang 8 sampai14 mikrometer, diluar ukuran tersebut gelombang infra red sensor tidak akan mendeteksinya atau tidak dapat terbaca oleh sensor tersebut. Pendeteksian sensor PIR  dapat dilakukan dengan gerakan tangan, dengan cara menggerakkan tangan ke arah sensor PIR.