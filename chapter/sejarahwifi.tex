Sejarah Perkembangan Wi-Fi (Wireless Fidelity)
Wireless Fidelity adalah satu standart wireless networking tanpa kabel. teknologi spesifikasi ini emiliki standart yang ditetapkan oleuh sebuah institusi internasional yang bernama IEEE  (Insitute of Electrical and Electronic Engineers). Di tahun 1997 sebuah lembagaindependen bernama IEEE membuat standart WLAN pertama yang diberi kode 802.11. dapat bekerja pada frekuensi 2,4GHz dengan kecepatan transfer data 2Mbps.
Empatsejarah singkat perkembangan protokol Wireless fidelity:
1. pada bulan juli 1999, IEEE merilis spesifikasi baru yang bernama 802.11b. dengan kecepatan transfer data maksimal11Mbps.
2. pada waktu yang hampir bersamaan institute of electrical and electronic engineers kembali membuat spesifikasi 802.11a yang menggunakan teknik berbeda. Frekuensi yang  digunakan 5GHz, dan mendukung kecepatan transfer data dengan kecepatan sampai 54Mbps
3.pada tahun 2002. institute of electrical and electronic engineers membuat spesifikasi baru yang dapat menggabungkan kelebihan antara 802.11b dengan 802.11a. spesifikasi baru  yang diberi kode 802.11g ini bekerja pada frekuensi 2,4GHz dengan kecpatan transfer data maksimal 54Mbps.
4.di tahun 2006 institute of electrical and electronic engineers mengembangkan teknologi terbarunya dengan menggabungkan teknologi 802.11b dengan 802.11g menjadi 802.11n. teknologi ini dikenal dengan istilah MIMO (Multiple Input Multiple Output) teknologi wireless fidelity terbaru
 
SEJARAH WI-FI
Sejarah Wi-Fi berasal dari sebuah istilah puluhan tahun Hi-Fi yang terdiri dari jenis output yang dihasilkan olehkualitas sound system. Teknollgi Wireless Fidelity berspesifikasi standart Institute of Electrical and Electronic Engineers atau yang disingkat dengan IEEE 802.. termasuk 802.11a, 802.11b, dan 802.11g. Wireless Fidelity atau yang disingkat dengan Wi-Fi adalah hanya istilah produk teknologi yang dipromosikan oleh WIFI Alliance.
Sejarah Wireless Fidelity itu sendiri dimulai ketika tahun 1985 dari hasil kerja keras insinyur Amerika dengan pengguna Teknologi penyebaran spektrum radio yang digunakan dalam Wi-Fi. Wireless LAN atau Wi-Fi dibuat dan tersedia untuk umum di Amerika Serikat di tahun 1985, tidak ada lisensi dari komisi komunikasi federal (FCC). Kemudian Michael Marcus mengusulkan untuk menggunakan wireless LAN dan teknologi radio untuk publik.

Wi-Fi adalah sebuah teknologi yang memanfaatkan peralatan teknologi untuk bertukar data menggunakan gelombang radio melalui jaringan komputer. Vic Hayes adalah penemu Wi-Fi yang kini dijuluki sebagai “ Father of Wi-Fi “. WI-Fi merupakan sekumpulan standar yang digunakan untuk Jaringan Lokal Nirkabel yang memiliki spesifikasi IEEE 802.11. Pengertian dari IEEE tersebut adalah sebuah organisasi internasional yang mempublikasikan beberapa persoalan kunci dari dunia networking komputer. Ada awalnya Wi-Fi hanya digunakan pada jaringan Lokal (LAN),seiring berjalannya waktu Wi-Fi dimanfaatkan masyarakat untuk mengakses internet. Penerapan Wi-Fi  ditujukan sebagai alternatif dari jaringan Lokal komputer LAN,dimana penggunaan kabel sudah tidak lagi effisien. Wi-Fi memiliki mobilitas yang tinggi,sehingga untuk mengakses WI-Fi ini tidak diperlukannya penyambung kabel untuk menghubungkan ke server.
Pada dasarnya,Wi-Fi terdiri dari sumber yang dihubungkan dengan access point melalui kabel backbone. Selanjutnya dipancarkan melalui gelombang elektromagnetik seperti pada LAN kabel biasa yang kemudian diterima oleh client (Contohnya PC desktop) melalui wireless adapter yang mendukung jaringan Wi-Fi berdasarkan standarisasi IEEE 802.11. Tetapi access point ini memiliki area yang sangat terbatas,500 feet (152.4 M) dalam ruangan tertutup dan 1000 feet (304.8 M) dalam ruangan terbuka.
Wi-Fi akan mengalami proses handoffs agar wireless client dapat melanjutkan komunikasi dengan server yang berbeda. Wireless client akan terus memonitor sinyal yang diterima oleh access point,jika kuat sinyal kurang dari nilai sensitivitas penerimaan (threshold) maka wireless akan melakukan handoffs yang selanjutnya akan mencari sinyal terdekat. Proses identifikasi dari wireless client untuk menemukan sinyal access point terkuat hanya dibatasi dalam waktu 60 second. Backbone search time adalah proses pencarian AP dan EP untuk dijadikan BSS. Untuk dapat berkomunikasi yang lama antara wireless client dengan access point harus memiliki level daya yang diterima di atas -77 dBm,jika kurang dari -77 dBm maka wireless client akan melakukan proses handoffs dengan beralih pada daya yang lebih tinggi dari access point sebelumnya.
Dibalik kelebihannya Wi-Fi yang sudah memiliki kebutuhan  akses internet yang lebih baik dibandingkan dengan akses internet yang menggunakan kabel,tetapi Wi-Fi masih memiliki beberapa kekurangan sekarang ini,diantaranya ada :
	Area coverage-nya yang sangat sempit,hanya dalam hitungan meter
	Hanya mencukupi akses internet dalam suatu daerah atau dalam ruangan saja
	Keamanan yang belum terjamin
	Membutuhkan banyak BTS untuk menjangkau seluruh area yang luas
	LoS (Line of Sight)

<<<<<<< HEAD
 Cara Kerja Wi-Fi
 Mode Akses Koneksi Wi-fi ada 2 yaitu :
1. AD-HOC
Sistem Ad-hoc adalah sistem peer to peer, dalam arti satu computer dihubungkan ke 1 computer dengan saling mengenal SSID. Bila digambarkan mungkin lebih mudah membayangkan sistem direct connection dari 1 computer ke 1 computer lainnya dengan mengunakan Twist pair cable tanpa perangkat HUB. Jadi terdapat 2 computer dengan perangkat WIFI dapat langsung berhubungan tanpa alat yang disebut access point mode. Pada sistem Adhoc tidak lagi mengenal system central (yang biasanya difungsikan pada Access Point). Sistem Adhoc hanya memerlukan 1 buah computer yang memiliki nama SSID atau sederhananya nama sebuah network pada sebuah card/computer. Dapat juga mengunakan MAC address dengan sistem BSSID (Basic Service Set IDentifier - cara ini tidak umum digunakan), untuk mengenal sebuah nama computer secara langsung. Mac Address umumnya sudah diberikan tanda atau nomor khusus tersendiri dari masing masing card atau 
perangkat network termasuk network wireless. Sistem Adhoc menguntungkan untuk pemakaian sementara misalnya hubungan network antara 2 computer walaupun disekitarnya terdapat sebuah alat Access Point yang sedang bekerja.
2. INFRASTRUKTUR
Sistem kedua yang paling umum adalah Infra Structure. Sistem Infra Structure membutuhkan sebuah perangkat khusus atau dapat difungsikan sebagai Access point melalui software bila mengunakan jenis Wireless Network dengan perangkat PCI card. Mirip seperti Hub Network yang menyatukan sebuah network tetapi didalam perangkat Access Point menandakan sebuah sebuah central network dengan memberikan signal (melakukan broadcast) radio untuk diterima oleh computer lain. Untuk mengambarkan koneksi pada Infra Structure dengan Access poin minimal sebuah jaringan wireless network memiliki satu titik pada sebuah tempat dimana computer lain yang mencari menerima signal untuk masuknya kedalam network agar saling berhubungan. Sistem Access Point (AP) ini  paling banyak digunakan karena setiap computer yang ingin terhubungan kedalam network dapat mendengar transmisi dari Access Point tersebut. Access Point inilah yang memberikan
 tanda apakah disuatu tempat memiliki jaringan WIFI dan secara terus menerus mentransmisikan namanya – Service Set IDentifier (SSID) dan dapat diterima oleh computer lain untuk dikenal. Bedanya dengan HUB network cable, HUB mengunakan cable tetapi tidak memiliki nama (SSID). Sedangkan Access point tidak mengunakan cable network tetapi harus memiliki sebuah nama yaitu nama untuk SSID. Contoh Wi-fi Hardware yang digunakan di masyarakat : Wi-fi dalam bentuk PCI Wi-fi dalam bentuk USB



=======


Penambahan materi tentang perbedaan wifi dengan wimax


Perbedaan antara WI-FI dengan WIMAX

Pada awalnya WI-FI dan WIMAX tidak memiliki banyak perbedaan, hanya perbedaan antara jarak jangkauan luas jaringan nya.
jika WI-FI hanya mampu menyalurkan sinyalnya hanya sampai beberapa meter saja dan semakin jauh jangkauan si pemakai WI-FI maka
semakin kecil pula sinyal yang diterimanya. Berbeda dengan WIMAX yang memiliki cakupan coverage area lebih luas atau jangkauan 
sinyalnya lebih luas.

\subsection{TeknikpelokalanWiFi}

Teknik pelokalan WiFi masuk dalam sejumlah kategori besar. Beberapa teknik estimasi lokasi mencoba Model propagasi sinyal secara 
langsung melalui ruang [Bahl dan Padmanabhan, 2000], dengan asumsi lokasi akses diketahui titik dan model atenuasi sinyal eksponensial. 
Namun, bahkan saat mempertimbangkan lokasi dan material dinding dan furnitur di dalam bangunan, keakuratan perambatan sinyal Modelnya 
sangat terbatas. Teknik lain mencoba model kemungkinan membaca berdasarkan lokasi spesifik [Haeberlen et al., 2004; Letchner et al., 
2005], mewakili kekuatan sinyal di lokasi yang diminati dengan distribusi probabilitas yang dipelajari dari data pelatihan Sedangkan 
lebih akurat dibanding propagasi sinyal model, metode ini secara inheren diskrit dan memiliki hanya kemampuan terbatas untuk interpolasi 
antar lokasi. Untuk mengatasi keterbatasan tersebut, Schwaighofer dan rekannya [2003] menunjukkan bagaimana menerapkan proses Gaussian 
ke lokalisasi kekuatan sinyal, menghasilkan model yang disediakan interpolasi melalui lokasi kontinu dengan pemodelan langsung
ketidakpastian dari data pelatihan [Ferris dkk, 2006] diperpanjang Teknik ini untuk lokalisasi WiFi dengan menggabungkan Model kekuatan 
sinyal GP dengan graph-based tracking, memungkinkan untuk lokalisasi yang akurat dalam skala skala besar. WiFiSLAM kami Teknik membangun 
model terbaru ini dan meluas mereka untuk kasus pemetaan dengan lokasi yang tidak diketahui \cite{}.

\subsection
{Jenis jenis Wireless}
1. Berbasis Ad-Hoc
Pada jaringan ini, komunikasi antara satu perangkat ke perangkat lain di lakukan secara
spontan atau langsung tanpa melalui konfigurasi tertentu selama Acces point masih dapat
diterima dengan baik oleh perangkat perangkat lain dalam jaringan ini
2. Berbasis Infrastruktur
Pada jaringan ini, satu atau lebih Acces Point menghubungkan jaringan WLAN melalui
jaringan berbasis kabel. Jadi pada jaringan ini, untuk melayani perangkat didalam
jaringan ini maka Acces Point memerlukan koneksi ke jaringan berbasis kabel terlebih
dahulu.
Karena banyak nya jenis jenis WLAN yang ada di pasaran, maka standar IEE 802.11
menetapkan antarmuka yang klien WLAN dengan Acces Point nya. Untuk membedakan
antara jariangan WLAN satu dengan jaringan WLAN lain nya, maka 802.11
menggunakan Service Set Identifier ( SSID ). Dengan penanda ini maka dapat dibedakan
jaringan WLAN satu dengan jaringan WLAN lain nya, sebab jaringan WLAN satu
dengan jaringan WLAN yang lain nya pasti memiliki nomor penanda SSID yang berbeda
pula. Acces Point menggunakan SSID untuk menentukkan lalu lintas paket data mana
yang di peruntukkan untuk Acces Point tersebut.
Standar 802.11 juga menentukkan frekuensi yang dapat di gunakan oleh jaringan WLAN.
Misal nya untuk industrial, scientific dan medical ( ISM) beroperasi pada freukensi radio
2,4GHz. 802.11 juga menentukkan tiga jenis tranmisi pada lapisan fisik untuk model
Open System Interconnection ( OSI ), yaitu direct-sequence spread spectrum ( DSSS ),
frecuency-hopping spread spectrum ( FHSS ), dan infrared. Selain pembagian frekuensi
di atas, standar 802.11 juga membagi frame nya menjadi tiga kategori, yaitu control, date
dan management.
Standar 802.11 membolehkan device ( perangkat ) mengikuti standar 802.11 untuk
berkomunikasi satu sama lain nya dengan kecepatan 1Mbps dan 2Mbps dalam jangkauan
kira kira 100 meter. Jenis lain dari standar 802.11 nanti di kembangkan untuk
menyediakan kecepatan transfer data yang lebih cepat dengan tingkat fungsionalitas yang
lebih baik dari yang ada saat ini. Saat ini terdapat beberapa jenis variant dari standar
802.11, yaitu 802.11a, 802.11b, dan 802.11g.
a. Standar 802.11a
Standar 802.11 digunakan untuk mendefinisikan jaringan wireless yang menggunakan
frekuensi 5 GHz Unlicensed National Information Infrastrusture ( UNII ). Kecepatan
jaringan ini lebih cepat dari standar 802.11 dan standar 802.11b pada kecepatan transfer
sampai 54Mbps. Kecepatan ini bisa lebih cepat lagi bisa menggunakan teknologi yang
tepat.
Untuk menggunakan standar 802.11a perangkat komputer ( device ) hanya memerlukan
dukungan kecepatan komunikasi 6Mbps, 12Mbps, dan 24Mbps. Standar 802.11
mengoperasikan channel atau saluran empat kali lebih banyak yang di lakukan standar
802.11 dan standar 802.11b. Walaupun standar 802.11a dan standar 802.11b memiliki
kesamaan lapisan media Acces Control ( MAC ), ternyata tetap tidak kompatibel dengan
standar 802.11 dan standar 802.11b karena standar 802.11a menggunakan frekuensi radio
5GHz sementara standar 802.11b hanya menggunakan frekuensi 2,4GHz.
Keunggulan dari standar 802.11a adalah karena beroperasi pada frekuensi 5GHz sehingga
tidak perlu bersaing dengan perangkat komunikasi tanpa kabel ( cordless ) lain nya seperti
telephone ( cordless phone ) yang biasa menggunakan frekuensi 2,4GHz.
b. Standar 802.11b
Standar 802.11b merupakan standar yang paling banyak di gunakan di kelas standar
802.11. Standar ini adalah pengembangan dari standar 802.11 dari lapisan fisik dengan
kecepatan tinggi. Standar 802.11b di gunakan untuk mendefiniskan jaringan wireless
direct-sequence spread spectrum ( DSSS ) yang menggunakan gelombang indusrial,
scientific and medical ( ISM ) 2,4GHz dan berkomunikasi pada kecepatan 11Mbps. Ini
lebih cepat di banding 1Mbps dan 2Mbps yang di tawarkan standar 802.11. Standar
802.11b juga lebih kompatibel dengan semua perangkat DSSS yang beroperasi pada

standar 802.11
Standar 802.11b hanya berkonsentrasi pada lapisan fisik dan media Acces Control ( MAC
). Standar ini hanya menggunakan satu jenis frame yang memiliki lebar maksimun 2,346
byte. Namun dapat di bagi menjadi 1,518 byte jika di hubungkan secara silang ( cross )
dengan perangkat acces point sehingga dapat juga berkomunikasi dengan jaringan
berbasis ethernet ( berbasis kabel ).
Standar ini hanya menekankan pada pengoperasian DSSS saja. Standar ini menyediakan
metode untuk perangkat perangkat tersebut untuk mencari ( discover ), asosiasi, dan
autentikasi satu sama lain. Standar ini juga menyediakan metode untuk menangani
tabrakan ( collision ) dan fragmentasi dan memungkinkan metode enkripsi melalui
protokol WEP ( wired equivalent protocol ).
c. Standar 802.11g
Standar ini pada dasar nya mirip dengan standar 802.11a yaitu menyediakan jalur
komunikasi kecepatan tinggi hingga 54Mbps. Namun, frekuensi yang di gunakan sama
dengan frekeunsi yang di gunakan standar 802.11 yaitu 2,4GHz dan juga dapat
kompatibel dengan standar 802.11b, yang tidak di miliki oleh standar 802.11a.
Standar 802.11g menggunakan modulasi OFDM untuk memperoleh kecepatan transfer
data berkecepatan tinggi. Tidak seperti perangkat standar 802.11a, perangkat standar
802.11g dapat berganti secara otomatis ke quadrature phase shift keying ( QPSK ) untuk
berkomunikasi dengan perangkat pada jaringan wireless yang menggunakan andar
802.11b.
Di bandingkan dengan standar 802.11a, ternyata standar 802.11g memiliki kelebihan
kompatibilitas dengan jaringan standar 802.11b. Namun, masalah yang sering muncul
adalah perangkat perangkat standar 802.11g yang mencoba berpindah ke jaringan standar
802.11b atau sebalik nya adalah masalah interfensi yang di akibatkan jaringan frekuensi
2,4GHz.
 
 Kesimpulan jenis jenis
 
Penggunaan teknologi jaringan berbasis wireless merupakan pilihan yang tepat saat ini. Hal ini disebebkan mulai bergesernya prilaku perusahaan dalam menjalankan bisnis mereka. Dengan portabilitas dan kompabilitas yang ditawarkan oleh teknologi wireless tentunya merupakan pilihan yang menarik. Namun dibalik itu harus dipertimbangin juga teknologi wireless apa yang tepat untuk diterapkan diperusahaan sehingga benar benar dapat membantu bisnis perusahaan tersebut. Hal ini dapat di lihat dari perbedaan masing masing standar wireless yang tersedia saat ini ( 802.11, 802.11a, 802.11b, 802.11g ). Di liat dari sisi keamanan, tentu nya 802.11b sedikit lebih baik karena dapat menerapkan enskripsi dengan menggunakan protokol WEP di dalam jaringan tersebut. Kalau diliat dari sisi tidak ada nya gangguan atau noise tentu nya teknologi 802.11a lebih unggul karena standar ini hanya menggunakan frekuensi 5GHz di mana frekuensi ini tidak banyak di gunakan oleh perangkat perangkat berbasis wireless lain nya. Sehingga untuk mengatasi masalah masalah tersebut, maka standar 802.11g muncul untuk menjembatani kelemahan pada standar sebelum nya. 

\subsection {Perkembangan}
Perkembangan teknologi perangkat komunikasi data melalui jaringan nirkabel atau Wireless LAN (WLAN) terus meningkat sejalan dengan penggunaan akses internet yang makin hari semakin banyak. Teknologi Wireless LAN yang direkomendasikan melalui standar IEEE 802.11 ada tiga, yaitu : Standar IEEE 802.11 (2,4 Ghz dengan kecepatan 2 Mbps), Standar IEEE 802.11a ( 5 GHz dengan kecepatan 5,4 Mbps), Standar IEEE 802.11b(2,4 GHz-2,5 GHz) dan Standar IEEE 802.11g ( 2,4 GHz dengan kecepatan 54 Mbps). Wireless fidelity atau yang sering kita kenal sebagai Wi-Fi merupakan teknologi WLAN dengan standar IEEE 802.11b yang beroperasi di frekuensi 2,4 GHz-2,5 GHz. Antena Access Point dalam stuktur jaringan WLAN mempunyai fungsi sebagai media yang mendistribusikan sinyal ke beberapa perangkat bergerak atau mobile station. Untuk meningkatkan kemampuan daya transmisi sinyal dan daya jangkauan pancaran gelombang elektromagnetik lebih jauh.   Untuk menunjang kemampuan tersebut dalam riset ini di rancang antena dasar bersifat susun array. Antena pada titik akses memiliki sifat directional. Sehingga antena dapat dirancang dengan model susun agar memperoleh gain yang lebih tinggi.  Antena susun dua patch terdistribusi melalui rangkaian transformer seperempat gelombang menggunkan model power divider T-Juntcion. Rangkaian transformer dirancang melalui saluran transmisi mikrostrip dengan struktur terdiri dari dua saluran keluaran dan satu saluran masuk yang memiliki nilai impedansi sama. Penempatan antar patch peradiasi secara linier satu sumbu koordinat dengan pengaturan jarak resonansi di atas seperempat gelombang pada titik pusat patch peradiasi. Material substrat PCB yang digunakan jenis duroid 5880 dengan ketebalan 1,57 mm dan konstanta dielektrik 2,2[3]. Untuk rancang bangun antena digunakan metode simulasi menggunakan perangkat lunak microwave office. Hasil rancang bangun antena susun dua patch diharapkan tercapai target parameter gain diatas 5 dB.  Teknologi mikrostrip merupakan sebuah medium yang sebutan lainnya substrate memiliki karakteristik dielektrik yang dapat digunakan untuk menghantarkan atau suatu propagasi gelombang elektromagnetik melalui teknologi MIC ( Microstrip Integrated Circiut ) untuk frekuensi gelombang mikro. Secara umum bentuk sebuah patch antena mikrostrip ada tiga, yaitu: persegi panjang, lingkaran dan ellips. Struktur dari antena mikrostrip, dimana lebar konduktor pada sisi permukaan atas substrat disebut patch.   Arah radiasi medan magnetic dari patch menuju pada lapisan substrat dengan ketebalan tertentu sampai bidang ground. Bidang ground merupakan lapisan konduktor yang menutupi seluruh lapisan substrat. Sehingga medan radiasi akan terpantul keseluruh permukaan substart dan sebagian menuju ke lapisan udara. 
>>>>>>> 6c8491a9f56fa9ed1281c59d9b13f480d314c113

Keamanan Wireless Wi-Fi

Jaringan Wifi memiliki lebih banyak kelemahan dibanding dengan jaringan kabel. Saat ini perkembangan
teknologi wifi sangat signifikan sejalan dengan kebutuhan sistem informasi yang mobile. Banyak
penyedia jasa wireless seperti hotspot komersil, ISP, Warnet, kampuskampus
maupun perkantoran sudah
mulai memanfaatkan wifi pada jaringan masing masing, tetapi sangat sedikit yang memperhatikan
keamanan komunikasi data pada jaringan wireless tersebut. Hal ini membuat para hacker menjadi tertarik
untuk mengexplore keamampuannya untuk melakukan berbagai aktifitas yang biasanya ilegal
menggunakan wifi.
Pada artikel ini akan dibahas berbagai jenis aktivitas dan metode yang dilakukan para hacker wireless
ataupun para pemula dalam melakukan wardriving. Wardriving adalah kegiatan atau aktivitas untuk
mendapatkan informasi tentang suatu jaringan wifi dan mendapatkan akses terhadap jaringan wireless
tersebut. Umumnya bertujuan untuk mendapatkan koneksi internet, tetapi banyak juga yang melakukan
untuk maksudmaksud
tertentu mulai dari rasa keingintahuan, coba coba, research, tugas praktikum,
kejahatan dan lain lain.
Beberapa Teknik Keamanan yang digunakan pada Wireless LAN
Dibawah ini beberapa kegiatan dan aktifitas yang dilakukan untuk mengamanan jaringan wireless :

Menyembunyikan SSID

Banyak administrator menyembunyikan Services Set Id (SSID) jaringan wireless mereka dengan maksud
agar hanya yang mengetahui SSID yang dapat terhubung ke jaringan mereka. Hal ini tidaklah benar,
karena SSID sebenarnya tidak dapat disembuyikan secara sempurna. Pada saat saat tertentu atau
khususnya saat client akan terhubung (assosiate) atau ketika akan memutuskan diri (deauthentication)
dari sebuah jaringan wireless, maka client akan tetap mengirimkan SSID dalam bentuk plain text
(meskipun menggunakan enkripsi), sehingga jika kita bermaksud menyadapnya, dapat dengan mudah
menemukan informasi tersebut. Beberapa tools yang dapat digunakan untuk mendapatkan ssid yang
dihidden antara lain, kismet (kisMAC), ssid_jack (airjack), aircrack ,
void11 dan masih banyak lagi.

Keamanan wireless hanya dengan kunci WEP

WEP merupakan standart keamanan & enkripsi pertama yang digunakan pada wireless, WEP memiliki
berbagai kelemahan antara lain :
● Masalah kunci yang lemah, algoritma RC4 yang digunakan dapat dipecahkan.
● WEP menggunakan kunci yang bersifat statis
● Masalah initialization vector (IV) WEP
● Masalah integritas pesan Cyclic Redundancy Check (CRC32)
WEP terdiri dari dua tingkatan, yakni kunci 64 bit, dan 128 bit. Sebenarnya kunci rahasia pada kunci
WEP 64 bit hanya 40 bit, sedang 24bit merupakan Inisialisasi Vektor (IV). Demikian juga pada kunci
WEP 128 bit, kunci rahasia terdiri dari 104bit. Seranganserangan
pada kelemahan WEP antara lain :
1. Serangan terhadap kelemahan inisialisasi vektor (IV), sering disebut FMS attack. FMS singkatan
dari nama ketiga penemu kelemahan IV yakni Fluhrer, Mantin, dan Shamir. Serangan ini
dilakukan dengan cara mengumpulkan IV yang lemah sebanyakbanyaknya.
Semakin banyak IV
lemah yang diperoleh, semakin cepat ditemukan kunci yang digunakan
(www.drizzle.com/~aboba/IEEE/rc4_ksaproc.pdf)
2. Mendapatkan IV yang unik melalui packet data yang diperoleh untuk diolah untuk proses
cracking kunci WEP dengan lebih cepat. Cara ini disebut chopping attack, pertama kali
ditemukan oleh h1kari. Teknik ini hanya membutuhkan IV yang unik sehingga mengurangi
kebutuhan IV yang lemah dalam melakukan cracking WEP.
3. Kedua serangan diatas membutuhkan waktu dan packet yang cukup, untuk mempersingkat waktu,
para hacker biasanya melakukan traffic injection. Traffic Injection yang sering dilakukan adalah
dengan cara mengumpulkan packet ARP kemudian mengirimkan kembali ke access point. Hal ini
mengakibatkan pengumpulan initial vektor lebih mudah dan cepat. Berbeda dengan serangan
pertama dan kedua, untuk serangan traffic injection,diperlukan spesifikasi alat dan aplikasi
tertentu yang mulai jarang ditemui di tokotoko,
mulai dari chipset, versi firmware, dan versi
driver serta tidak jarang harus melakukan patching terhadap driver dan aplikasinya.
