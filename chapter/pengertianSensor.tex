\section {Sensor}

\subsection{Pengertian Sensor}

Sensor merupaka sebuah alat yang digunakan untuk mendeteksi sesuatu seperti suhu,cahaya, kecepatan, jarak, dan lain-lain. dan sensor juga dapat berfungsi untuk 
mengukur suatu besaran (magnitude). menurut pendapat lain, sensor adalah suatu alat yang dapat mengubah dari besaran fisika menjadi besaran listrik. pada masa sekarang ini sensor telah dibuat dengan ukuran yang kecil sehingga dapat memudahkan para penggunanya untuk menggunakan sensor dan juga dapat meminimalsir pemakaian tempat dalam penciptaan sebuah alat.


\subsection{Jenis-jenis Sensor}
	\begin{itemize}
	\item Sensor Cahaya a.k.a. Light Sensors
	\item Sensor Tekanan a.k.a. Preasure Sensors
	\item Sensor Jarak a.k.a. Proximity Sensors
	\item Sensor Ultrasonik
	\item Sensor Kecepatan a.k.a. RPM Sensors
	\item Sensor Magnet
	\item Sensor Penyandi a.k.a. Encoder Sensors
	\item Sensor Suhu a.k.a. Temperature Sensosrs
	\item Flow Meter Sensors
	\item Sensor Api a.k.a. Flame Sensors
	\end{itemize}

\subsection{Light Sensors}
Dari namanya saja sudah teretebak untuk apa sensor ini. Light Sensors digunakan untuk objek yang memiliki warna dan/atau cahaya, yang bisa diubah mejadi daya yang beracam variasinya. Sensor cahaya terdiri dari 3 macam kategori diantaranya:

	\begin{itemize}
	\item Fotovolataic, sebuah proses kerja dari sensor yang mengubah energi sinar langsung menjadi energi listrik, dikarenakan penyinaran tersebut akan menghasilkan tegangan listrik dari berpindahnya elektron.
	\item Fotokonduktif, prinsip sensor ini memberikan perubahan pertahanan pada sel-selnya, dengan prinsip kerja semakin tinggi intensitas cahaya yang diterima maka akan semakin kecil nilai tahanannya
	\item Fotolistrik, sensor yang kinerjanya berdasarkan pantulan sinar, dari posisi atau jaraknya ataupun target pemantulannya. Terdiri dari pasangan sumber cahaya dan penerima.
	\end {itemize}

\subsection{Contoh Sensor}
	\begin {itemize}
	\item Light Depedendent Resistor, sensir yang berfungsi untuk mengubah intensitas cahaya menjadi hambatan listrik. dengan kinerja dari Light Dependent Resistor yaitu, semakin tinggi intesitas cahaya yang mengenai perukan Light Dependent Resistor, senakin besar juga hambatan listrik yang terjadi, begitu pun sebaliknya. Light Dependent Resistor ini dapat dipakai dalam pembuatan lampu otomatis. Lampu akan hidup sendirinya saat matahari terbenam karena intesitas cahaya yang terbaca oleh sensor sangatlah sedikit. Sebaliknya pula saat cahaya yang terbaca oleh sensor banyak maka lampu akan menyala.