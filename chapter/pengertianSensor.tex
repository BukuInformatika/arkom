\section {Sensor}

\subsection{Pengertian Sensor}

Sensor merupaka sebuah alat yang digunakan untuk mendeteksi sesuatu seperti suhu,cahaya, kecepatan, jarak, dan lain-lain. dan sensor juga dapat berfungsi untuk 
mengukur suatu besaran (magnitude). menurut pendapat lain, sensor adalah suatu alat yang dapat mengubah dari besaran fisika menjadi besaran listrik. pada masa sekarang ini sensor telah dibuat dengan ukuran yang kecil sehingga dapat memudahkan para penggunanya untuk menggunakan sensor dan juga dapat meminimalsir pemakaian tempat dalam penciptaan sebuah alat.


\subsection{Jenis-jenis Sensor}
	\begin{itemize}
	\item Sensor Cahaya a.k.a. Light Sensors
	\item Sensor Tekanan a.k.a. Preasure Sensors
	\item Sensor Jarak a.k.a. Proximity Sensors
	\item Sensor Ultrasonik
	\item Sensor Kecepatan a.k.a. RPM Sensors
	\item Sensor Magnet
	\item Sensor Penyandi a.k.a. Encoder Sensors
	\item Sensor Suhu a.k.a. Temperature Sensosrs
	\item Flow Meter Sensors
	\item Sensor Api a.k.a. Flame Sensors
	\end{itemize}

\subsection{Light Sensors}
Dari namanya saja sudah teretebak untuk apa sensor ini. Light Sensors digunakan untuk objek yang memiliki warna dan/atau cahaya, yang bisa diubah mejadi daya yang beracam variasinya. Sensor cahaya terdiri dari 3 macam kategori diantaranya: