\section {Sensor}

\subsection{Pengertian Sensor}

Sensor merupaka sebuah alat yang digunakan untuk mendeteksi sesuatu seperti suhu,cahaya, kecepatan, jarak, dan lain-lain. dan sensor juga dapat berfungsi untuk 
mengukur suatu besaran (magnitude). menurut pendapat lain, sensor adalah suatu alat yang dapat mengubah dari besaran fisika menjadi besaran listrik. pada masa sekarang ini sensor telah dibuat dengan ukuran yang kecil sehingga dapat memudahkan para penggunanya untuk menggunakan sensor dan juga dapat meminimalsir pemakaian tempat dalam penciptaan sebuah alat.


\subsection{Jenis-jenis Sensor}
	\begin{itemize}
	\item Sensor Cahaya a.k.a. Light Sensors
	\item Sensor Tekanan a.k.a. Preasure Sensors
	\item Sensor Jarak a.k.a. Proximity Sensors
	\item Sensor Ultrasonik
	\item Sensor Kecepatan a.k.a. RPM Sensors
	\item Sensor Magnet
	\item Sensor Penyandi a.k.a. Encoder Sensors
	\item Sensor Suhu a.k.a. Temperature Sensosrs
	\item Flow Meter Sensors
	\item Sensor Api a.k.a. Flame Sensors
	\end{itemize}

\subsection{Light Sensors}
Dari namanya saja sudah teretebak untuk apa sensor ini. Light Sensors digunakan untuk objek yang memiliki warna dan/atau cahaya, yang bisa diubah mejadi daya yang beracam variasinya. Sensor cahaya terdiri dari 3 macam kategori diantaranya:

	\begin{itemize}
	\item Fotovolataic, sebuah proses kerja dari sensor yang mengubah energi sinar langsung menjadi energi listrik, dikarenakan penyinaran tersebut akan menghasilkan tegangan listrik dari berpindahnya elektron.
	\item Fotokonduktif, prinsip sensor ini memberikan perubahan pertahanan pada sel-selnya, dengan prinsip kerja semakin tinggi intensitas cahaya yang diterima maka akan semakin kecil nilai tahanannya
	\item Fotolistrik, sensor yang kinerjanya berdasarkan pantulan sinar, dari posisi atau jaraknya ataupun target pemantulannya. Terdiri dari pasangan sumber cahaya dan penerima.
	\end {itemize}

\subsection{Contoh Sensor}
	\begin {itemize}
	\item Light Depedendent Resistor, sensor yang berfungsi untuk mengubah intensitas cahaya menjadi hambatan listrik. dengan kinerja dari Light Dependent Resistor yaitu, semakin tinggi intesitas cahaya yang mengenai perukan Light Dependent Resistor, senakin besar juga hambatan listrik yang terjadi, begitu pun sebaliknya. Light Dependent Resistor ini dapat dipakai dalam pembuatan lampu otomatis. Lampu akan hidup sendirinya saat matahari terbenam karena intesitas cahaya yang terbaca oleh sensor sangatlah sedikit. Sebaliknya pula saat cahaya yang terbaca oleh sensor banyak maka lampu akan menyala.
	\item Fotodiode, sensor ini berfungsi mengubah intensitas cahaya menjadi konduktivitas dioda. Prinsip kerja fotodiode, energi pancaran cahaya jatuh pada titik pertemuan dan menyebabkan sebuah elektron bebas dan hole. Contoh produk yang menggunakan fotodiode misalnya Line Tracer dengan kinerja sensor menerima input warna yang berbeda dari objek garis yang dipantulkan oleh pancaran LED, sehingga Line Tracer melaju cepat dan melewati garis dengan tepat.
	\item Fototransistor, sebuah sensor yang kerjanya hampir mirip dengan saklar, dengan perbedaan pada denyut masuk ke dalam basis. Pada transistor biasa denyut yang diberikan berupa arus DC, pada fototrasnsistor sendiri denyut yang diberikan pada basis berupa intensitas cahaya yang cocok dengan karakteristik fototransistor tersebut. Keadaan normal sensor ini adalah pada kolektor mendapat reverse bias, dan emitor mendapat reward bias. Kaki kolektor pada sensor akan selalu mengalami arus bocor, dan terjadi di antara kolektor dan basis.Selain dipengaruhi temperature sensor juga dipengaruhi oleh intensitas cahaya yang jatuh pada pengosongan antara kolektor dan basis.
	\end{itemize}

\subsection{Sensor Tekanan}
Sensor tekanan sensor ini memiliki transduser yang mengukur ketegangan kawat, dimana mengubah tegangan mekanis menjadi sinyal listrik. Dasar penginderaannya pada perubahan tahanan pengantar (transduser) yang berubah akibat perubahan panjang dan luas penampangnya. Contoh produk yang menggunakan sensor Tekanan, seperti: Alat untuk mendeteksi tekanan darah orang dewasa secara otomatis. Alat tersebut dilakukan dengan manset yang dipasang di lengan pasien, kemudian dipompa sampai pada tekanan tertentu yang selanjutnya baru dilakukan pengukuran tekanan darah.
	
\subsection{Sensor Magnet}
	\begin	{itemize}
	\item Sensor magnet dapat digunakan dalam penggunaan pintu garasi otomatis. Sensor ini menggunakan sensor magnet MIKROKONTROLERA AT89S51. Dalam perancangannya,rangkaian terdiri dari beberapa block diagram yang memiliki fungsi berbeda. Sumber tegangan berfungsi pengaktif komponen,sumber tegangan yang keluar dari trafo dimasukkan ke dalam regulator yang berfungsi sebagai pengubah tegangan. Block input berfungsi sebagai pemberi masukan kepada alat pintu garasi otomatis. Input pada rangkaian pintu garasi otomatis adalah magnet yang direspon sensor magnet. Jika sensor 1 dan 2 merespon adanya magnet,maka rangkaian ini dapat berfungsi. Block proses berfungsi untuk pengolah masukan,dimana masukan yang diterima akan diolah berdasarkan yang telah deprogram dalam MIKROKONTROLERA AT89S51. Output rangkaian ini adalah pergerakan motor. 
	\item Cara kerjanya adalah arus yang masuk ke diode bridge dari sumber tegangan diserahkan terlebih dahulu. Setelah itu,arus masuk  ke IC 7805 yang berungsi untuk memecah arus menjadi +5 volt. Untuk menjalankan alat ini pertama kita harus mendekatkan magnet ke dekat sensor yang terhubung ke port 2.1. Jika sensor merespon adanya magnet,maka sensor akan terhubung secara langsung ke port 2.1 yang bernilai 0. Kemudian,led yang berada dalam optocopler yang dialiri arus +5 volt akan menyala. Nyala dari led dalam optocopler diterima oleh fototransistor yang berada di dalam optocopler berfungsi untuk mengubar efek cahaya menjadi sinyal listrik.
		  Motor stepper yang terhubung ke port 2.0 akan bergerak sesuai dengan jumlah nilai yang dimasukkan pada register 0 yaitu,sebanyak 30h. Dan garasi akan tertutup apabila sensor yang lainnya atau yang terhubung ke port 2.0 terkena magnet. Dan untuk membukanya kembali maka sensor yang harus terkena pertama kali adalah yang terhubung ke port 2.0,jika kita mendekatkan magnet ke sensor yang terhubung  ke port 2.1 maka pintu tidak akan terbuka.
	\end{itemize}
	
\subsection{Sensor Suhu}
	\begin {itemize}
	\item Sensor suhu DS1621  adalah termometer digital dan termoslat memiliki resolusi output sebesar 9 bit. Alarm panas aktif ketika suhu panas dari peralatan melebihi suhu yang telah diatur (TH). Alarm panas akan terus aktif sampai suhu panas turun pad suhu yang telah diatur (TL). Beberapa dari keistimewaan sensor suhu DS1621 ,yaitu seperti mengukur suhu dari -550C sampai 1250C ,mampu membaca suhu hingga 9bit,dan disupply mulai dari 2,7V sampai 5,5V. Sensor ini memiliki jangkauan pengukuran suhu antara -55°C hingga +125°C dengan keakurasian ±0,5°C. Tegangan outpoutnya adalah 10mV/°C. tegangan output dapat langsung terhubung dengan salah satu port mikrokontroler yang memiliki kemapuan ADC,yaitu ATmega8535. ADC pada ATmega8535 beresolusi 10bit dapat memberikan keluaran 210=1024. Apabila menggunakan daya 5V,resolusi menghasilkan daya 5000mV/1024 = 4.8mV. Karena LM35 memiliki resolusi output 10mV/°C,maka resolusi termometer yang dibuat dengan ATmega8535 adalah 10mV/4.8mV ~ 0.5°C. Data yang dikeluarkan oleh sensor ini akan naik sebanyak 10mV setiap derajat celcius sehingga diperoleh persamaan sebagai berikut :
				V = suhu °C x 10mV
	\end{itemize}