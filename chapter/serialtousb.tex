\section{Jenis - Jenis chipset Serial to USB}
\subsection{Definisi}
\subsubsection{Motherboard}
Motherboard adalah salah satu komponen, atau perangkat keras komputer yang berfungsi  untuk papan induk dan menyambungkan seluruh komponen bagian-bagian CPU menjadi satu kesatuan sistem komputer. Motherboard sendiri  mempunyai  bentuk seperti  papan PCB yang biasa digunakan untuk keperluan perangkat elektronik, dimana di dalamnya terdapat banyak socket untuk menghubungkan keseluruhan sistem komputer.
\subsubsection{Chipset}
Chipset adalah kumpulan - kumpulan dari chip ditambah IC yang dibuat untuk dapat bekerja dalam komputer atau perangkat elektronik lainnya.
Chipset sendiri lebih banyak dipakai sebagai input dan output pada alat elektronik. jadi jika sebuah board atau tempat chipset dipasang mengalami kerusakan, maka kemungkinan besar chipset yang bermasalah.

\subsubsection{I. PENGKABELAN PORT SERIAL}

Pada prinsipnya, serial ialah pengiriman data dilakukan per bit, sehingga lebih lambat dibandingkan parallel seperti pada port printer yang mampu mengirim 8 bit sekaligus dalam sekali detak. Beberapa contoh serial ialah mouse, scanner dan system akuisisi data yang terhubung ke port COM1/COM2.

\subsubsection{Pengkabelan}
Device pada serial port dibagi menjadi 2 (dua ) kelompok yaitu Data Communication Equipment (DCE) dan Data Terminal Equipment (DTE). Contoh dari DCE ialah modem, plotter, scanner dan lain lain sedangkan contoh dari DTE ialah terminal di komputer.

Spesifikasi elektronik dari serial port merujuk pada Electronic Industry Association (EIA) :

1. “Space” (logika 0) ialah tegangan antara + 3 hingga +25 V.

2. “Mark” (logika 1) ialah tegangan antara –3 hingga –25 V.

3. Daerah antara + 3V hingga –3V tidak didefinisikan /tidak terpakai

4. Tegangan open circuit tidak boleh melebihi 25 V.

5. Arus hubungan singkat tidak boleh melebihi 500mA.

\subsubsection{USB}
USB ialah port yang sangat diandalkan saat ini dengan bentuknya yang kecil dan kecepatan datanya yang tinggi. Anda dapat menghubungkan hingga 127 produk usb dalam 1 komputer. USB versi 1.1 mendukung 2 kecepatan yaitu mode kecepatan penuh 12Mbits/s dan kecepatan rendah 1.5 Mbits/s. USB 2.0 mempunyai kecepatan 480Mbits/s yang dikenal sebagai mode kecepatan tinggi. 
Saat ini transfer data menggunakan port USB sudah semakin marak, port USB menjadi pilihan utama karena ukuran yang ringkas dan kecepatan transfer data yang cukup besar.  Sebagai perbandingan, Bus  PCI saat ini mendukung transfer data hingga 132 MB/s, dimana AGP (pada 66MHz) mendukung hingga 533 MB/s. AGP  dapat melakukan ini karena kemampuannya untuk mentransfer data pada ujung naik dan turun detak 66MHz. 
