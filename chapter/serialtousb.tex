\section{Jenis - Jenis chipset Serial to USB}
\subsection{Definisi}
\subsubsection{Motherboard}
Motherboard dapat  terpasang di berbagai jenis slot untuk menghubungkan dengan perangkat keras atau komponen-komponen komputer seperti prosesor, RAM, kartu grafis, Hard Drive atau hardisk, DVD Drive, Keyboard, mouse dan lain-lain. Setiap Motherboard memiliki kumpulan chip dan yang mengendalikannya disebut sebagai chipset. Ketika sebuah motherboard baru dikembangkan, motherboard tersebut umumnya telah menggunakan chipset baru. Motherboard adalah salah satu komponen, atau perangkat keras komputer yang berfungsi  untuk papan induk dan menyambungkan seluruh komponen bagian-bagian CPU menjadi satu kesatuan sistem komputer. Motherboard sendiri  mempunyai  bentuk seperti  papan PCB yang biasa digunakan untuk keperluan perangkat elektronik, dimana di dalamnya terdapat banyak socket untuk menghubungkan keseluruhan sistem komputer.
\subsubsection{Chipset}
Chipset adalah kumpulan - kumpulan dari chip ditambah IC yang dibuat untuk dapat bekerja dalam komputer atau perangkat elektronik lainnya.
Chipset sendiri lebih banyak dipakai sebagai input dan output pada alat elektronik. jadi jika sebuah board atau tempat chipset dipasang mengalami kerusakan, maka kemungkinan besar chipset yang bermasalah.

Berikut ini adalah fungsi dan juga peran dari chipset northbridge:

Mengendalikan komunikasi yang terjalin antara processor, RAM, AGP, PCI Express, dan juga southbridge.
Dapat melakukan pengendalian terhadap video.
Memiliki peran yang sangat penting dalam menentukan jumlah, tipe dan juga kecepatan dari sebuah CPU atau processor yang dihubungkan ke dalam motherboard.
Menentukan jumlah, kecepatan dan juga tipe pada RAM yang bisa digunakan pada motherboard.

\subsubsection{I. PENGKABELAN PORT SERIAL}

Pada prinsipnya, serial ialah pengiriman data dilakukan per bit, sehingga lebih lambat dibandingkan parallel seperti pada port printer yang mampu mengirim 8 bit sekaligus dalam sekali detak. Beberapa contoh serial ialah mouse, scanner dan system akuisisi data yang terhubung ke port COM1/COM2.

\subsubsection{Pengkabelan}
Device pada serial port dibagi menjadi 2 (dua ) kelompok yaitu Data Communication Equipment (DCE) dan Data Terminal Equipment (DTE). Contoh dari DCE ialah modem, plotter, scanner dan lain lain sedangkan contoh dari DTE ialah terminal di komputer.

Spesifikasi elektronik dari serial port merujuk pada Electronic Industry Association (EIA) :

1. “Space” (logika 0) ialah tegangan antara + 3 hingga +25 V.

2. “Mark” (logika 1) ialah tegangan antara –3 hingga –25 V.

3. Daerah antara + 3V hingga –3V tidak didefinisikan /tidak terpakai

4. Tegangan open circuit tidak boleh melebihi 25 V.

5. Arus hubungan singkat tidak boleh melebihi 500mA.

\subsubsection{USB}
USB ialah port yang sangat diandalkan saat ini dengan bentuknya yang kecil dan kecepatan datanya yang tinggi. Anda dapat menghubungkan hingga 127 produk usb dalam 1 komputer. USB versi 1.1 mendukung 2 kecepatan yaitu mode kecepatan penuh 12Mbits/s dan kecepatan rendah 1.5 Mbits/s. USB 2.0 mempunyai kecepatan 480Mbits/s yang dikenal sebagai mode kecepatan tinggi. 
Saat ini transfer data menggunakan port USB sudah semakin marak, port USB menjadi pilihan utama karena ukuran yang ringkas dan kecepatan transfer data yang cukup besar.  Sebagai perbandingan, Bus  PCI saat ini mendukung transfer data hingga 132 MB/s, dimana AGP (pada 66MHz) mendukung hingga 533 MB/s. AGP  dapat melakukan ini karena kemampuannya untuk mentransfer data pada ujung naik dan turun detak 66MHz.

USB Universal Serial Bus adalah singkatan dari Universal Serial Bus. USB merupakan suatu teknologi yang memungkinkan kita untuk menghubungkan alat eksternal (peripheral) seperti scanner, printer, mouse, papan ketik (keyboard), alat penyimpan data (zip drive), flash disk, kamera digital atau perangkat lainnya ke komputer kita. USB sangat mendukung transfer data sebesar 12 Mbps ( juta bit per detik). Komputer (PC) saat ini, umumnya sudah memiliki port USB. Biasanya disediakan minimal 2 port. Jika dibandingkan dengan paralel port dan serial port, penggunaan port USB lebih mudah dalam penggunaannya.

perangkat Super­Speed, seperti hard disk dari Buffalo dan Western Digital, tetapi kecepatan transfernya hanya 100 MB/detik. SSD jauh lebih cepat dengan 250 MB/detik. SSD pertama dengan USB 3.0 dan kecepatan di atas 300 MB/detik telah diumumkan dan rencananya akan dipasarkan pada kuartal ketiga tahun ini.
 Kelebihan USB 3.0 :
 1. tidak diperlukannya pasokan daya tambahan untuk penggunaan hard disk eksternal.
 2. suplai daya untuk perangkat USB 2.0 maksimal 500 mA, sementara untuk perangkat 3.0 mencapai 900 mA.
 3. Hal yang baru di USB 3.0, host-controller akan terhubung ke perangkat yang tepat ketika akan menuliskan data pada USB flashdisk. Sementara itu, USB 2.0 mengirim data ke semua perangkat de­ngan cara polling – tak soal apakah perangkat tersebut tepat atau tidak.

Flash disk merupakan salah satu perangkat yang menggunakan USB port untuk menghubungkannya dengan komputer. Flash disk berfungsi sebagai media penyimpanan data. Cara menghubungkan Flash disk ke komputer sangat mudah. Masukkan flash disk tersebut ke port USB yang telah tersedia. Jika komputer kita menggunakan Windows XP maka secara otomatis Flash disk tersebut akan dikenali. Sedangkan jika kita menggunakan windows 98 atau windows 9x maka biasanya driver Flash tersebut harus terlebih dahulu di install. Setelah Flash disk

\subsubsection{Kabel USB}
Jika dibuka, kabel USB akan terlihat ada 4 warna, yaitu merah, coklat, kuning dan biru. Kabel berwarna merah dan coklat berfungsi sebagai power / untuk arus listrik. Kabel berwarna kuning dan biru berfungsi untuk membawa / mentransfer data.
\subsubsection{Cara Menghubungkan USB Flash disk dengan Komputer }
Flash disk merupakan salah satu perangkat yang menggunakan USB port untuk menghubungkannya dengan komputer. Flash disk berfungsi sebagai media penyimpanan data. Cara menghubungkan Flash disk ke komputer sangat mudah. Masukkan flash disk tersebut ke port USB yang telah tersedia. Jika komputer kita menggunakan Windows XP maka secara otomatis Flash disk tersebut akan dikenali. Sedangkan jika kita menggunakan windows 98 atau windows 9x maka biasanya driver Flash tersebut harus terlebih dahulu di install. 
RS232 adalah standard komunikasi serial yang digunakan untuk koneksi periperal ke periperal. Biasa juga disebut dengan jalur I/O ( input / output ). Contoh yang paling sering kita temui adalah koneksi antara komputer dengan modem, atau komputer dengan mouse bahkan bisa juga antara komputer dengan komputer, semua biasanya dihubungkan lewat jalur port serial RS232. 
Standar ini menggunakan beberapa piranti dalam implementasinya. Paling umum yang dipakai adalah plug / konektor DB9 atau DB25. Untuk RS232 dengan konektor DB9, biasanya dipakai untuk mouse, modem, kasir register dan lain sebagainya, sedang yang konektor DB25, biasanya dipakai untuk joystik game.


menambahkan bandwitdh hingga 480 Mbit/s [60 MB/s] (disebut “Hi-Speed”). hasil modifikasi dari Engineering Change Notices (ECN). beberapa kemampuan yng ditambahkan ECN dpat dilihat di USB.org: Battery Charging Specification 1.1 (memungkinkan charge perankat ke usb misal kamera digital/ handphone), Micro-USB Cables and Connectors Specification 1.01 (telah support dengan port us ukuran mikro seperti pada kameradigital/ handphone), Link Power Management Addendum ECN (memungkinkan USB dalam kondisi Slepp saat tidak digunakan)

\subsubsection{Tipe USB}
Mini-USB adalah konektor standar untuk perangkat-perangkat mobile. Mini-USB lebih kecil ketimbang USB reguler, dan digunakan di perangkat kecil seperti smartphone, atau kamera digital. Mini-USB kini sudah ditinggalkan dan sebagian besar kini menggunakan micro-USB.
Micro-USB
Pembenahan bentuk dari Mini-USB, dan telah disepakati oleh banyak pabrikan untuk memproduksi berbagai perangkat mobile-nya dengan USB tipe ini. Bentuk USB inilah yang terdapat pada charger smartphone kita. Sebagai tambahan, hanya Apple yang tak menggunakan USB tipe ini.
If you need to connect a serial device to your computer but your computer doesn't have a serial port, then you need a USB to Serial adapter. Finding the best adapter can be a challenge since there are so many differnt models, types and manufacturers. These 5 steps will help you make the right decision. 

\subsubsection{Konektor USB Tipe A}
Sebagian besar Kabel USB memiliki konektor USB Tipe A di satu sisi, karena Konektor USB Tipe A yang berbentuk persegi panjang inilah yang dipasangkan pada bagian komputer. PC atau Komputer Personal pada umumnya memiliki beberapa port USB Tipe A. Perangkat-perangkat seperti keyboard dan mouse juga menggunakan konektor USB Tipe A di satu sisinya. Beberapa jenis adaptor yang dipergunakan oleh perangkat-perangkat elektronik portabel juga memiliki port USB Tipe A untuk melakukan pengisian ulang baterainya. Nama resmi konektor USB Tipe A ini adalah USB Standard A.

\subsubsection{Konektor USB Tipe B}
Nama resmi konektor USB Tipe B ini adalah USB Standard B. Konektor USB Tipe B berbentuk bujur sangkar dengan sedikit lekukan di kedua sudut atas. Jenis Konektor USB Tipe B ini biasanya digunakan oleh Printer ataupun Scanner dan tidak sepopuler USB Tipe-A.

\subsubsection{USB}
USB Tipe C yang baru dirilis beberapa waktu lalu memiliki bentuk yang sama dengan tipe B (port dan konektor). Port untuk tipe C berukuran 8.4 mm dan 2.6 mm. Tentunya, dengan ukuran yang kecil akan kompatibel untuk device kecil. Kabel yang berada pada bagian ujung sangat kecil sehingga dapat digunakan secara bolak-balik
USB Universal Serial Bus adalah singkatan dari Universal Serial Bus. USB merupakan suatu teknologi yang memungkinkan kita untuk menghubungkan alat eksternal (peripheral) seperti scanner, printer, mouse, papan ketik (keyboard), alat penyimpan data (zip drive), flash disk, kamera digital atau perangkat lainnya ke komputer kita. USB sangat mendukung transfer data sebesar 12 Mbps ( juta bit per detik). Komputer (PC) saat ini, umumnya sudah memiliki port USB. Biasanya disediakan minimal 2 port. Jika dibandingkan dengan paralel port dan serial port, penggunaan port USB lebih mudah dalam penggunaannya.

\subsection{Serial}
\subsubsection{Serial}
Serial port merupakan salah satu sarana yang digunakan oleh sebuah PC untuk berhubungan dengan perangkat luar. Melalui port ini, semua peralatan luar yang terkoneksi dengan PC dapat dikontrol atau dikendalikan dengan memberikan sebuah perintah melalui PC. Ada dua cara dalam komunikasi serial, yaitu komunikasi data serial secara sinkron dan komunikasi data serial secara asinkron. Pada komunikasi data serial secara sinkron, sinyal clock dikirimkan bersama-sama dengan data serial. Sedangkan komunikasi data secara asinkron, sinyal clock tidak dikirim bersama-sama dengan data serial, melainkan
dibangkitkan sendiri-sendiri oleh rangkaian penerima data (receiver) maupun rangkaian pengirim data (transmitter)

Pada PC, serial port yang digunakan termasuk jenis asinkron, dimana komunikasi serial ini dikerjakan oleh UART (Universal Asynchronous Receiver / Transmitter). IC UART ini berfungsi untuk mengubah data parallel menjadi data serial dan menerima data serial yang kemudian diubah kembali menjadi data parallel.
Standar sinyal komunikasi serial yang banyak digunakan adalah standar RS232 yang dikembangkan oleh Electronic Industry Association and the Telecommunications Industry Association (EIA / TIA). Standar ini hanya menyangkut komunikasi data antara komputer (Data Terminal Equipment – DTE) dengan alat – alat pelengkap komputer (Data Circuit-Terminating Equipment – DCE). Standarad RS232 inilah yang biasa digunakan pada serial port IBM PC Compatibel
Pada UART, kecepatan pengiriman data (baudrate) dan fase clock pada sisi transmitter dan pada sisi receiver harus sinkron. Untuk itu diperlukan diperlukan sinkronisasi antara transmitter dan receiver. Hal ini dilakukan oleh bit ‘Start’ dan bit ‘Stop’.

RS232 adalah standard komunikasi serial yang digunakan untuk koneksi periperal ke periperal. Biasa juga disebut dengan jalur I/O ( input / output ). Contoh yang paling sering kita temui adalah koneksi antara komputer dengan modem, atau komputer dengan mouse bahkan bisa juga antara komputer dengan komputer, semua biasanya dihubungkan lewat jalur port serial RS232. 
Standar ini menggunakan beberapa piranti dalam implementasinya. Paling umum yang dipakai adalah plug / konektor DB9 atau DB25. Untuk RS232 dengan konektor DB9, biasanya dipakai untuk mouse, modem, kasir register dan lain sebagainya, sedang yang konektor DB25, biasanya dipakai untuk joystik game.

Standar RS232 ditetapkan oleh Electronic Industry Association and Telecomunication Industry Association pada tahun 1962. Nama lengkapnya adalah EIA/TIA-232 Interface Between Data Terminal Equipment and Data Circuit-Terminating Equipment Employing Serial Binary Data Interchange.
Port Serial RS232 juga mempunyai fungsi yaitu untuk menghubungkan / koneksi dari perangkat yang satu dengan perangkat yang lain, atau peralatan standart yang menyangkut komunikasi data antara komputer dengan alat-alat pelengkap komputer.

\subsubsection{Serial Interface}
Sebuah saluran data yang mentransfer data digital secara serial: satu bit demi satu di atas satu kawat atau serat. Serial Interface mungkin memiliki beberapa baris, tetapi hanya satu baris digunakan untuk data. Garis lain yang digunakan untuk kontrol.
