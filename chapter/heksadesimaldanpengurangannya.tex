\section{BILANGAN HEKSADESIMAL}

Bilangan heksadesimal yang hanya berbasis 16 memiliki nilai yang dapat disimbolkan dengan 0, 1, 2, 3, 4, 5, 6, 7, 8, 9, A, B, C, D, E, F.
Munculnya bilangan heksadesimal pada operasi komputasi karena apabila operasi bilangan biner untuk data yang lumayan besar akan menjadi 
sulit untuk dibaca, sehingga bilangan heksa selalu digunakan pada saat menggambarkan memori dan instruksi. digunakannya bilang heksa sebagai 
pengganti bilangan biner dikarenakan setiap digit bilangan heksadesimal dapat mewakili 4 bit bilangan binerdan 2 digit bilangan heksa dapat 
mewakili 1 byte 

	Pengurangan pada bilangan heksadesimal

melakukan pengurangan secara berurutan dimulai dari digit yang paling kanan, jika bilangan yang dikurangi lebih sedikit atau lebih kecil 
dari bilangan pengurang maka otomatis bilangan pengurang akan melakukan pinjaman 1 ke bilangan sebelummnya. contoh: 
	
	FBC(16) - 321(16) = ..........(16)
 Langkah-langkah penyelesaian: 
FBC 
3 2 1 ----- (-)

C - 1 = 12 -1 = 11, hasil pengurangan adalah B 
B - 2 = 11 - 2 = 9,  hasil pengurangan adalah 9
F - 3 = 15 - 3 = 12, hasil pengurangan adalah C 

=======
\section{BILANGAN HEKSADESIMAL}

Bilangan heksadesimal yang hanya berbasis 16 memiliki nilai yang dapat disimbolkan dengan 0, 1, 2, 3, 4, 5, 6, 7, 8, 9, A, B, C, D, E, F.
Munculnya bilangan heksadesimal pada operasi komputasi karena apabila operasi bilangan biner untuk data yang lumayan besar akan menjadi 
sulit untuk dibaca, sehingga bilangan heksa selalu digunakan pada saat menggambarkan memori dan instruksi. digunakannya bilang heksa sebagai 
pengganti bilangan biner dikarenakan setiap digit bilangan heksadesimal dapat mewakili 4 bit bilangan binerdan 2 digit bilangan heksa dapat 
mewakili 1 byte 

	Pengurangan pada bilangan heksadesimal

melakukan pengurangan secara berurutan dimulai dari digit yang paling kanan, jika bilangan yang dikurangi lebih sedikit atau lebih kecil 
dari bilangan pengurang maka otomatis bilangan pengurang akan melakukan pinjaman 1 ke bilangan sebelummnya. contoh: 
	
	FBC(16) - 321(16) = ..........(16)
 Langkah-langkah penyelesaian: 
FBC 
3 2 1 ----- (-)

C - 1 = 12 -1 = 11, hasil pengurangan adalah B 
B - 2 = 11 - 2 = 9,  hasil pengurangan adalah 9
F - 3 = 15 - 3 = 12, hasil pengurangan adalah C 


Jadi hasil pengurangannya adalah  FBC(16) - 321(16) = C9B(16) 