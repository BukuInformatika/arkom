/section{definisi}
Unshielded twisted pair (UTP) adalah jenis kabel tembaga mana-mana yang digunakan pada kabel telepon dan jaringan area lokal (LAN).
Dalam operasi garis seimbang, kedua kabel membawa sinyal sama dan berlawanan, dan tujuan mendeteksi perbedaan antara keduanya. Hal ini dikenal sebagai modetransmission differential. Sumber kebisingan mengenalkan sinyal ke kabel dengan menggabungkan medan listrik atau medan magnet dan cenderung berpasangan dengan kedua kabel dengan sama. Kebisingan tersebut menghasilkan sinyal mode umum yang dibatalkan pada receiver saat sinyal perbedaan diambil. Metode ini mulai gagal saat sumber suara dekat dengan kabel sinyal; kawat yang lebih dekat akan berpasangan dengan noise lebih kuat dan penolakan mode umum pada penerima akan gagal untuk menghilangkannya. Masalah ini terutama terlihat pada kabel telekomunikasi dimana pasangan kabel yang sama saling berdekatan satu sama lain selama bermil-mil. Satu pasang bisa menyebabkan crosstalkin lain dan itu aditif sepanjang kabel. Memutar pasang counter efek ini seperti pada masing-masing setengah memutar kawat yang terdekat dengan sumber noise yang dipertukarkan. Menyediakan sumber yang mengganggu tetap seragam, atau hampir, lebih dari jarak satu putaran, kebisingan yang diinduksi akan tetap umum. Diferensial sinyal juga mengurangi radiasi elektromagnetik dari kabel, bersamaan dengan atenuasi yang terkait sehingga memungkinkan jarak yang lebih jauh antara pertukaran. Tingkat twist (juga disebut pitch of twist, biasanya didefinisikan dalam twists per meter) merupakan bagian dari spesifikasi untuk jenis kabel tertentu. Bila pasangan di dekatnya memiliki tingkat twist yang sama, konduktor yang sama dari pasangan yang berbeda dapat berulang kali berbohong satu sama lain, dan sebagian membatalkan manfaat dari mode diferensial. Untuk alasan ini biasanya ditentukan bahwa, setidaknya untuk kabel yang mengandung sejumlah kecil pasang, tingkat twist harus berbeda. Berbeda dengan twisted pair terlindung atau tergoda (biasanya perisai kabel F / UTP atau S / FTP), kabel UTP (unshielded twisted pair) tidak dikelilingi oleh perisai. UTP adalah tipe kabel utama untuk telepon dan sangat umum untuk jaringan komputer, terutama sebagai kabel patch atau koneksi jaringan sementara karena tingginya fleksibilitas kabel.

/subsection{sejarah}
Telepon paling awal menggunakan jalur telegraf, atau kawat kisi-kisi belakang kawat tunggal. Pada tahun 1880-an, trem listrik dipasang di banyak kota, yang menyebabkan kebisingan masuk ke sirkuit ini. Tuntutan hukum yang tidak layak, perusahaan telepon beralih ke sirkuit seimbang, yang memiliki manfaat sampingan pengurangan emisi, sehingga meningkatkan jangkauan. Karena distribusi tenaga listrik menjadi lebih umum, ukuran ini terbukti tidak memadai. Dua kabel, digantung di kedua sisi palang palang di tiang listrik, berbagi rute dengan jalur listrik. Dalam beberapa tahun, meningkatnya penggunaan listrik kembali membawa peningkatan interferensi, sehingga para insinyur merancang sebuah metode yang disebut transposisi kawat, untuk membatalkan interferensi tersebut. Dalam transposisi kawat, posisi pertukaran kabel setiap beberapa tiang. Dengan cara ini, kedua kabel akan menerima similarEMI dari kabel listrik. Ini merupakan implementasi awal memutar, dengan tingkat twist sekitar empat twists perkilometre, atau enam per mil. Garis seimbang kawat terbuka semacam itu dengan transposisi periodik masih bertahan sampai sekarang di beberapa daerah pedesaan. Kabel twisted-pair diciptakan oleh Alexander Graham Bell pada tahun 1881. [3] Pada tahun 1900, jaringan saluran telepon Amerika secara keseluruhan adalah twisted pair atau open wire dengan transposisi untuk mencegah gangguan. Saat ini, sebagian besar dari jutaan pasang twisted pair di dunia adalah sambungan telepon rumah, yang dimiliki oleh perusahaan telepon, digunakan untuk layanan suara, dan hanya ditangani atau bahkan dilihat oleh pekerja telepon.
Unshielded twisted pair (UTP) kabel ditemukan di banyak jaringan Ethernet dan sistem telepon. Bagian khas dari warna-warna ini (putih / biru, biru / putih, putih / oranye, oranye / putih) muncul di sebagian besar kabel UTP. Kabel biasanya dibuat dengan kabel tembaga yang diukur pada 22 atau 24 American Wire Gauge (AWG), [4] dengan insulasi berwarna yang biasanya dibuat dari isolator seperti polietilen atau FEP dan total paket yang dilapisi jaket apolyetilena. Untuk kabel telepon luar kota yang berisi ratusan atau ribuan pasang, kabelnya terbagi menjadi beberapa paket kecil namun identik. Setiap bundel terdiri dari pasangan twisted yang memiliki tingkat twist berbeda. Bundel pada gilirannya dipelintir bersama untuk membuat kabel. Pasangan yang memiliki tingkat twist yang sama di dalam kabel masih bisa mengalami beberapa derajatcrosstalk. Pasangan kawat dipilih dengan hati-hati untuk meminimalkan crosstalk dalam kabel besar. Kabel twisted pair unshielded dengan tingkat twist yang berbeda Kabel UTP juga merupakan kabel yang paling umum digunakan dalam jaringan komputer. ModernEthernet, standar jaringan data yang paling umum, bisa menggunakan kabel UTP. Kabel twisted pair sering digunakan pada jaringan data untuk koneksi jarak pendek dan menengah karena biaya yang relatif lebih rendah dibandingkan kabel serat optik dan koaksial. UTP juga menemukan peningkatan penggunaan dalam aplikasi video, terutama di kamera keamanan. Banyak kamera termasuk keluaran UTP dengan terminal kecil; Bandwidth kabel UTP telah meningkat agar sesuai dengan sinyal baseband dari televisi. Karena UTP adalah saluran transmisi yang seimbang, balun diperlukan untuk terhubung ke peralatan yang tidak seimbang, misalnya menggunakan konektor BNC dan dirancang untuk kabel koaksial.

/subsection{ATP terhadap UTP}
ATP dilepaskan dari sebagian besar jenis sel dan berfungsi sebagai molekul pensinyalan ekstraselular melalui aktivasi anggota dua keluarga besar reseptor P2X dan P2Y. Meskipun tiga reseptor P2Y mamalia telah dikloning yang diaktifkan secara selektif oleh nukleotida urida, demonstrasi langsung pelepasan UTP seluler belum dilaporkan. Studi farmakologi reseptor P2Y4 yang diekspresikan pada sel astrocytoma manusia 1321N1 mengindikasikan bahwa reseptor ini diaktifkan oleh UTP namun tidak oleh ATP. Stimulasi mekanis sel 1321N1 juga menghasilkan pelepasan molekul yang secara nyata mengaktifkan reseptor P2Y4 yang dinyatakan. Nukleotida ini terbukti UTP dengan dua cara. Pertama, analisis kromatografi cair kinerja tinggi medium dari sel 3321] yang dilepaskan oleh O3PO4 memuat 1321N1 sehingga stimulasi mekanis menghasilkan peningkatan besar pada spesies radioaktif yang digerakkan bersama dengan UTP asli. Spesies ini terdegradasi oleh inkubasi dengan apyrase pyrophosphohydrolase nonspesifik atau dengan heksokinase dan secara khusus hilang melalui inkubasi dengan enzim UTP-glukosa pirofosforilase UDP. Kedua, uji sensitif yang mengkuantifikasi massa UTP pada konsentrasi nanomolar rendah dirancang berdasarkan spesifisitas nukleotida dari pirofosforilase UDP-glukosa. Dengan menggunakan uji ini, rangsangan mekanis sel 1321N1 terbukti menghasilkan peningkatan tingkat UTP medium dari 2,6 menjadi 36,4 pmol / 106cells dalam 2 menit. Peningkatan ini disejajarkan dengan augmentasi serupa dari tingkat ATP ekstraselular. Metode quenching fluoresensi berbasis calcein digunakan untuk memastikan bahwa tidak ada peningkatan kadar nukleotida medium yang dapat dipertanggungjawabkan dengan lisis sel. Secara bersamaan, hasil ini secara langsung menunjukkan pelepasan UTP yang diinduksi secara mekanis dan menggambarkan kopling yang efisien dari pelepasan ini ke aktivasi reseptor P2Y4.
